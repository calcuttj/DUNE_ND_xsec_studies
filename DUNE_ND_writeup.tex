\setlength{\headheight}{15pt}
\documentclass[12pt]{article}
\usepackage{fancyhdr}
\lhead{}
\chead{}
\rhead{}
\renewcommand{\headrulewidth}{0pt}
\pagestyle{fancy}
\usepackage{graphicx}
\usepackage[top=2cm,bottom=3cm]{geometry}
\usepackage[svgnames]{xcolor}
\usepackage[colorlinks=true,linkcolor=DarkBlue,citecolor=DarkBlue]{hyperref}
\usepackage{xspace}
\usepackage{rotating}
\usepackage{units}
%\usepackage{subfig}
%\usepackage{amssymb, amsmath}
\usepackage{amsmath}
\usepackage{authblk}
\usepackage{lineno}
\usepackage{listings} 
\usepackage[normalem]{ulem}
%\usepackage{placeins}
\usepackage[section]{placeins}

\usepackage{SIunits}
\usepackage{hepunits}
\usepackage{hepparticles}
\usepackage{cancel}
\usepackage{hepnames}
\usepackage{epstopdf}
\usepackage{mathtools}
\usepackage{caption}
\usepackage[aboveskip=-10pt]{subcaption}
\usepackage[capitalise]{cleveref}
\usepackage{braket}
\usepackage{slashed}

\newcommand{\todo}[1]{{\color{red} TODO: #1}}
\newcommand\red[1]{{\color{red}#1}}
\newcommand{\ccpi}{CC1$\pi^0$\xspace}
\newcommand{\ccpis}{CC$\pi^0$\xspace}
\newcommand{\ccpip}{CC1$\pi^+$\xspace}
\newcommand{\ncpi}{NC1$\pi^0$\xspace}
\newcommand{\ccqe}{CCQE\xspace}
\newcommand{\mares}{\ensuremath{M_A^\mathrm{res}}\xspace}
\newcommand{\ppi}{\ensuremath{|\mathbf{p}_{\pi^0}|}\xspace}
\newcommand{\mb}{MiniBooNE\xspace}
\newcommand{\minerva}{MINER\ensuremath{\nu}A\xspace}
\newcommand{\neut}{\textsc{neut}\xspace}
\newcommand{\nuance}{\textsc{nuance}\xspace}
\newcommand{\tmu}{\ensuremath{T_{\mu}}\xspace}
\newcommand{\pmu}{\ensuremath{|\textbf{p}_{\mu}|}\xspace}
\newcommand{\cost}{\ensuremath{\cos{\theta_{\mu}}}\xspace}
\newcommand{\enu}{\ensuremath{E_{\nu}}\xspace}
\newcommand{\qq}{\ensuremath{Q^{2}}\xspace}
\newcommand{\qqqe}{\ensuremath{Q^{2}_{\textrm{QE}}}\xspace}
\newcommand{\pf}{\ensuremath{p_{F}}\xspace}
\newcommand{\eb}{\ensuremath{E_{b}}\xspace}
\newcommand{\carb}{C\ensuremath{^{12}}\xspace}
\newcommand{\oxy}{O\ensuremath{^{16}}\xspace}
\newcommand{\ie}{i.e.\xspace}
\newcommand{\eg}{e.g.\xspace}
\newcommand{\ma}{\ensuremath{M_{\textrm{A}}}\xspace}
\newcommand{\maqe}{\ensuremath{M_{\textrm{A}}^{\textrm{QE}}}\xspace}
\newcommand{\numu}{\Pnum}
\newcommand{\nue}{\Pnue}
\newcommand{\numubar}{\APnum}
\newcommand{\nuebar}{\APnue}
\newcommand{\enuqerfg}{\ensuremath{E^{\nu}_{\textrm{QE,RFG}}}\xspace}
\newcommand{\enuqe}{\ensuremath{E^{\nu}_{\textrm{QE}}}\xspace}
\newcommand{\chisq}{\ensuremath{\chi^{2}}\xspace}
\newcommand{\chisqmin}{\ensuremath{\chi^{2}_{\textrm{min}}}\xspace}
\newcommand{\chtwo}{CH\ensuremath{_{2}}\xspace}
\newcommand{\wroclaw}{Wroc{\l}aw\xspace}
\newcommand{\km}{\kilo\meter\xspace}
\newcommand{\m}{\meter\xspace}
\newcommand{\evsq}{\eV\ensuremath{^{2}}\xspace}
\newcommand{\POD}{P{\O}D\xspace}
\newcommand{\ecal}{ECal\xspace}
\newcommand{\ecals}{ECals\xspace}
\newcommand{\dsecal}{Ds-ECal\xspace}
\newcommand{\vol}[4]{\ensuremath{#1\times#2\times\unit{#3}{#4}}\xspace}
\newcommand{\area}[3]{\ensuremath{#1\times\unit{#2}{#3}}\xspace}
\newcommand{\pizero}{\pi^{0}\xspace}
\newcommand{\kg}{\kilo\gram\xspace}
\newcommand{\lep}{\ell}
\newcommand{\mnn}{multi-nucleon--neutrino\xspace}
\newcommand{\elt}{\ensuremath{E_{<}}\xspace}
\newcommand{\egt}{\ensuremath{E_{>}}\xspace}

\renewcommand\Im{\operatorname{Im}}

\graphicspath{{figures/}}

\newif\ifpdf
\ifx\pdfoutput\undefined
   \pdffalse
\else
   \pdfoutput=1
   \pdftrue
\fi
\ifpdf
   \usepackage{graphicx}
   \usepackage{epstopdf}
   %\DeclareGraphicsRule{.eps}{pdf}{.pdf}{`epstopdf #1}
   \pdfcompresslevel=9
\else
   \usepackage{graphicx}
\fi

\graphicspath{{figs/}}

\title{Current Status and Future Progess of DUNE ND studies}

\date{}
\begin{document}


\author[1]{Jake Calcutt}
\author[1]{Joshua Hignight}
\author[1]{Kendall Mahn}
\affil[1]{Michigan State University}
%\author[2]{Joshua Hignight}
%\author[3]{Kendall Mahn}


\maketitle
\thispagestyle{fancy}
%\linenumbers
%\begin{abstract}
%This report is a review of the current implementation of cross section related sources of systematic uncertainty for the DUNE Near Detector Taskforce (ND TF), charged with evaluating three possible near detector configurations. It identifies critical sources of systematic uncertainty, some of which are already covered, suggests studies to further improve the current uncertainty implementation, and summarizes future improvements to the systematic uncertainty outside the scope of the ND TF. 
%\end{abstract}
%A clear statement of what the committee deems to be an ideal (practical) scenario would be an excellent start. We can bring that to VALOR to determine what they can implement, and then ask the committee for some feedback on the limitations of the final VALOR parameterization to put in the report along with their full recommendations. The full simulation and analysis chain should not die with the task force, and the final report should include recommendations for what should be studied beyond the TF timeline.

\section{Overview}\label{sec:view}

%This document serves as a writeup detailing the current status of the DUNE ND studies and the progress that has been made so far, as well as a plan on how to further the studies in the near future. 

The Deep Underground Neutrino Experiment (DUNE) is a next-generation Long Baseline neutrino experiment aimed to achieve current scientific goals set out by the High Energy Physics community. It consists of both a Near and Far Detector separated by 1300km and standing in the NuMI beam created at Fermilab\cite{DUNE_CDR1}. While the design of the Liquid Argon Far Detector has been finalized, there is still ongoing effort in deciding the configuration of the Near Detector. The main near detector design includes a Fine-Grained Tracker (FGT) with possible inclusion of an upstream detector - being either a Liquid (LArTPC) or High Pressure Gaseous Argon TPC (GArTPC). DUNE's goals will require systematic uncertainties in the interaction model to be below the 2\% limit after a near-to-far extrapolation\cite{DUNE_review}.  The focus of the work described in this document is then to quantify the abilities of the standalone FGT detector and additional LAr/GAr TPC to achieve this limit in the near-to-far extrapolation. The sufficiency of current kinematic parameterization to handle model variations is also considered. 
%\begin{enumerate}
%\item Review the available material describing the handling of cross-section uncertainties within the DUNE Near Detector Task Force in the (\url{http://docs.dunescience.org:8080/cgi-bin/ShowDocument?docid=1291}{VALOR TN} sections 3 and 4 and the material presented at the 10am CT Thu July 14 ND Physics Working Group Meeting).
%\item Recommend changes within the existing framework that would better describe the current level of uncertainty in neutrino-nucleus interaction physics. These recommendations should be prioritized w.r.t. how crucial they are to the extraction of oscillation parameters (especially CP violation) and should take into account the limited timescale and manpower of the the NDTF (initial report Sept 2016, final report March 2017 ? about 0.5 FTE of cross-section resources).
%\item Make suggestions for studies that would help confirm the assumptions and/or results of the NDTF, should the time and manpower for these studies become available.
%\item  Make recommendations for a long term strategy for DUNE to study ND capabilities to i) measure neutrino-nucleus interactions and ii) constrain oscillation physics systematic uncertainties,  beyond the scope of the task force (task force charge can be found at this \url{https://docs.google.com/presentation/d16yY5CzRwo_243jpBzhVVeunvxYjIb7FzoS745RwURgw/edit#slide=id.p7}{link}.
%\end{enumerate}

\section{Motivation, Methods, and Results}
As the analysis techniques for DUNE are developed, checks on the cross section model are necessary. Variations arise in the different handling of Final State Interaction (FSI) effects by Monte Carlo event generators, as well as choice of Near Detector configuration. Sets of neutrino and antineutrino events on Argon-40 are produced with the GENIE\footnote[1]{GENIE version 2.10, RFG}\cite{GENIE}, NEUT\footnote[2]{NEUT version 5.3.6, Nieves et. al RPA+2p2h/MEC $M_A$ = 1.01}\cite{NEUT}, and NuWro\footnote[3]{NuWro version (FIND THE VERSION) LFG + RPA + Nieves et. al}\cite{NUWRO} according to the 2015 DUNE CDR fluxes. The various data sets are passed through the NUISANCE\cite{NUISANCE} software to reduce the various outputs to a common format, in turn saving all final state particle information for each event. Each particle is than randomly accepted or rejected according to efficiencies for each detector. Currently, only a robust description of the FGT efficiency is available, and simple thresholds for protons are applied for the GAr\footnote[4]{~100 MeV/c} and LAr ND and FD\footnote[5]{~200 MeV/c}. Currently, we do not have the efficiencies for $\mu$, $\pi^{+,-}$, and $\pi^{0}$ in the LAr and GAr, and so the detectors are assumed to be perfectly efficient (no rejections) to these particles. $\pi^0$ efficiency information is also currently missing in the FGT configuration, and so are assumed fully accepted.
\begin{figure}[h]
\minipage{.5\textwidth}
\includegraphics[width=\linewidth]{numu_ND_flux.png}
\endminipage
\minipage{.5\textwidth}
\includegraphics[width=\linewidth]{trkeff_proton.png}
\endminipage
\caption{Left: $\nu_\mu$ flux at DUNE ND used to generate events, Right: Tracking efficiency for protons in FGT.}
\end{figure}
%\newline

\section{Reconstructed Energy}\label{sec:Reco}

%\subsection{Final State, Reconstructed Energy}

A framework for investigating the differences in reconstructed energy calculations between different generators and ND configurations has been developed. %However, because it is calculated only by summing the final state energy (generator truth information), this is largely incomplete, and it serves only as a starting point for more robust studies. The full studies will include detector efficiencies and kinematic cuts to fully display the differences between models and configurations. 

The current development of this work includes a calculation of the final state energy by summing the total energy from final state leptons and pions (all charges) and the kinetic energy of final state protons after passing efficiencies through the data sets. All neutrons are assumed undetectable. 
\begin{equation}
E_{reco} = \Sigma {E_{lep}} + \Sigma E_{\pi} + \Sigma (E_{prot} - M_{prot})
\end{equation}
This is achieved by using the NUISANCE software package to reduce the output from various Monte Carlo neutrino event generators (including GENIE, NEUT, NUWRO, as well as the nuclear reaction and transport simulation software, GiBuu) into a common format that can easily be analyzed. 

%More sophisticated calculations of reconstructed energy, based on the type of detector design being considered, will be developed, and particle tracking and PID efficiencies for each detector will be included as this information becomes available.
Work has begun to include tracking/PID efficiencies to accept or reject final state particles. So far, only the full information (barring $\pi^0$ efficiencies) for the FGT is available, but the inclusion of the LAr or GAr TPCs will be easily implemented. 
% * <calcuttj@msu.edu> 2017-02-27T16:22:43.284Z:
%
% ^.

%As a first check in the ability to reconstruct neutrino energy, I began investigating the differences between the total final state energy and initial state energy. I've verified with Callum that the NUISANCE framework successfully treats the particle stacks from the various generators, and any differences come from how the different generators handle FSI. 



\subsection{Neutron Multiplicities \& Energy}

Differences between models in the number of final state neutrons and the total energy into FS neutrons can largely affect reconstruction of neutrino energy. Large variations in reconstructed energy can arise due to missing energy caused by the inability to detect neutrons in the various models.  

To investigate this, we have looked at GENIE, NEUT, and NUWRO to see if the different models showed a large difference in the neutron energy and multiplicity.  
This was done for CCQE-like, CC1$\pi$, 2P2H, and everything else (``Other'') interactions and neutrinos as well as anti-neutrinos.  
In all cases, the generators agreed rather well with each other even though there were some differences in the neutron multiplicity. 
These differences only account for a small fraction of the events.
Figure~\ref{fig:Neutron_multi_2p2h_ND} shows an example of this for 2P2H neutrino events, while the other interaction modes can be found in Appendix~\ref{app:Neutron_Multiplicities}
  
\begin{figure}
\centering
\begin{subfigure}[b]{0.32\textwidth}
  \includegraphics[width=\textwidth]{nneutrons_v_total_ene/Nneutrons_Total_ENe_2p2h_GENIE_ND_numu.pdf}
\end{subfigure}
\begin{subfigure}[b]{0.32\textwidth}
  \includegraphics[width=\textwidth]{nneutrons_v_total_ene/Nneutrons_Total_ENe_2p2h_NEUT_ND_numu.pdf}
\end{subfigure}
\begin{subfigure}[b]{0.32\textwidth}
  \includegraphics[width=\textwidth]{nneutrons_v_total_ene/Nneutrons_Total_ENe_2p2h_NUWRO_ND_numu.pdf}
\end{subfigure}
\caption{The neutron multiplicity vs total neutron energy for 2P2H interactions for GENIE, NEUT, and NUWRO, respectively.  Even though they do show a different phase-space for the neutron multiplicity, they all agree where most of the energy lost to neutrons should be.  This is similar for other interaction types as well.}
\label{fig:Neutron_multi_2p2h_ND}
\end{figure}

Further more, the difference in the multiplicities between the generators becomes irrelevant after a ND to FD extraction, as can be seen in Figure~\ref{fig:Neutron_multi_2p2h_ND_FD}.
Here, the region where the most energy is lost agrees very well between the ND and FD for all generators and interaction types.
The areas with low statistics do show a disagreement, but very few events fall into this area.

\begin{figure}
\centering
\begin{subfigure}[b]{0.32\textwidth}
  \includegraphics[width=\textwidth]{nneutrons_v_total_ene/Nneutrons_Total_ENe_2p2h_GENIE_ND_FD_numu_norm.pdf}
\end{subfigure}
\begin{subfigure}[b]{0.32\textwidth}
  \includegraphics[width=\textwidth]{nneutrons_v_total_ene/Nneutrons_Total_ENe_2p2h_NEUT_ND_FD_numu_norm.pdf}
\end{subfigure}
\begin{subfigure}[b]{0.32\textwidth}
  \includegraphics[width=\textwidth]{nneutrons_v_total_ene/Nneutrons_Total_ENe_2p2h_NUWRO_ND_FD_numu_norm.pdf}
\end{subfigure}
\caption{The ratio of the ND to the FD for neutron multiplicity vs total neutron energy for 2P2H interactions for GENIE, NEUT, and NUWRO, respectively.  In the area where the largest amount of energy is lost to neutrons, low multiplicity and low energy, the agreement between the ND and FD is almost perfect.} 
\label{fig:Neutron_multi_2p2h_ND_FD}
\end{figure}

Care should still be taken when calculating the total neutrino of the event if only a calorimetric approach is used, as is the case in this document.
This is because as much as 50\% of the energy can be taken away by the neutron in CCQE-like events.
Even for 2P2H events, as shown in Figure~\ref{fig:Neutron_multi_ene_enu_2p2h_ND}, it can be as much as 30\%.
The other interaction modes and anti-neutrino events can also be found in Appendix~\ref{app:Neutron_Multiplicities} and have a smaller fraction.

\begin{figure}
\centering
\begin{subfigure}[b]{0.32\textwidth}
  \includegraphics[width=\textwidth]{nneutrons_ene_enu/Nneutrons_Enu_true_2p2h_GENIE_ND_numu_norm.pdf}
\end{subfigure}
\begin{subfigure}[b]{0.32\textwidth}
  \includegraphics[width=\textwidth]{nneutrons_ene_enu/Nneutrons_Enu_true_2p2h_NEUT_ND_numu_norm.pdf}
\end{subfigure}
\begin{subfigure}[b]{0.32\textwidth}
  \includegraphics[width=\textwidth]{nneutrons_ene_enu/Nneutrons_Enu_true_2p2h_NUWRO_ND_numu_norm.pdf}
\end{subfigure}
\caption{Neutron multiplicity vs total neutron energy divide by the neutrino energy for 2P2H interactions from GENIE, NEUT, and NUWRO, respectively.  For low multiplicity, the neutron carries away a significant fraction of the neutrino energy.} 
\label{fig:Neutron_multi_ene_enu_2p2h_ND}
\end{figure}


 Additionally, investigations in the ability for errors in GENIE to cover the differences between models have been started. 

\subsection{Particle Multiplicity and Momentum}

Nucleon multiplicity and momentum distributions offer similar information as do N vs. E distributions, while specifically looking at protons and charged pions along with detector thresholds can enlighten the ability of the detectors' reconstruction capabilities. For these studies, the 3-momentum of the final state (after efficiencies) protons or charged pions are summed. The magnitude is then plotted against the multiplicity for the specific particle type. 

%\begin{equation}
%|\vec{P_{total,i}}| = |\Sigma \vec{P_i}|
%\end{equation}

\section{Parameterization}
\subsection{$Q^2$ studies}

The first study to be considered is the sufficiency of a purely $Q^2$ parameterization (a la VALOR) in treating the uncertainties for CCQE. To be considered is the relative change between the models (displayed as a single ratio of Other/GENIE separately at the near and far detectors) and this change as it is extrapolated between the near and far detector (as a double ratio of Other/GENIE,Near/Far).
%\newline
\begin{figure}[h]
\minipage{.5\textwidth}
\includegraphics[width=\linewidth]{Q2_ratios/CCQE_NEUT_GENIE_numu_near_Q2.png}
\endminipage
\minipage{.5\textwidth}
\includegraphics[width=\linewidth]{Q2_ratios/CCQE_NEUT_GENIE_numu_NF_Q2.png}
\endminipage
\caption{$Q^2$ distributed events using DUNE flux, no efficiencies applied. Left: Ratio of NEUT to GENIE output at ND, Right: Double ratio of NEUT to GENIE, Near to Far. Of note is the relative flatness in the low-$Q^2$ region in the double ratio, becoming less flat toward the higher end.}
\end{figure}
%\newline
\begin{figure}[h]
\minipage{.5\textwidth}
\includegraphics[width=\linewidth]{eff_Q2_ratios/CCQE_NEUT_GENIE_numu_near_Q2.png}
\endminipage
\minipage{.5\textwidth}
\includegraphics[width=\linewidth]{eff_Q2_ratios/CCQE_NEUT_GENIE_numu_NF_Q2.png}
\endminipage
\caption{$Q^2$ distributed events using DUNE flux, no efficiencies applied. Left: Ratio of NEUT to GENIE output at ND (high $Q^2$ region out of bounds of plot), Right: Double ratio of NEUT to GENIE, Near to Far. Flatness is generally lost throughout double ratio, particularly made worse in higher end.}
\end{figure}
\FloatBarrier

\subsection{$q_0 vs. q_3$ studies}

In comparison to the purely $Q^2$ parameterization, a simple $q_0 vs. q_3$ parameterization was also considered. One concern is the possibility of inconsistency between the two parameterizations in variations between models. 
%\newline
\begin{figure}[h]
\minipage{.5\textwidth}
\includegraphics[width=\linewidth]{q0_q3_ratios/CCQE_NEUT_GENIE_numu_near_q3_q0.png}
\endminipage
\minipage{.5\textwidth}
\includegraphics[width=\linewidth]{q0_q3_ratios/CCQE_NEUT_GENIE_numu_NF_q3_q0.png}
\endminipage
\caption{q3 vs. q0 distributed events using DUNE flux. Left: Ratio of NEUT to GENIE output at ND, Right: Double ratio of NEUT to GENIE, Near to Far}
\end{figure}
%\newline
\begin{figure}[h]
\minipage{.5\textwidth}
\includegraphics[width=\linewidth]{eff_q0_q3_ratios/CCQE_NEUT_GENIE_numu_near_q3_q0.png}
\endminipage
\minipage{.5\textwidth}
\includegraphics[width=\linewidth]{eff_q0_q3_ratios/CCQE_NEUT_GENIE_numu_NF_q3_q0.png}
\endminipage
\caption{q3 vs. q0 distributed events using DUNE flux. Left: Ratio of NEUT to GENIE output at ND, Right: Double ratio of NEUT to GENIE, Near to Far. Much of the FD has a tighter phase space due to being generator-level truth information, so a good deal of the ND-with-efficiency phase space is lost. Within the FD-covered phase space in the double ratio, flatness seems to be lost in the lower region, and much more bin-to-bin variation is present in the upper region (possibly need more statistics?).}
\end{figure}
\FloatBarrier

\subsection{Nucleon multiplicity vs. W}

Differences in mapping from Ereco to true variables can arise from shape differences in Nucleon multiplicities vs. W distributions. The first foray into these studies have been started, and need to be fleshed out for meaningful results. 

\section{Future Work}\label{sec:Future}

The above studies need to be furthered and expanded upon to successfully arrive at useful conclusions on ND configuration choice. 

\begin{itemize}


\item Extend studies to include LAr and GAr efficiency information when available
%\item Current uncertainties are invariant of target choice. Need to extend studies to include Calcium-40 target for FGT studies
\item Map from Ereco, yreco into true variables and investigate differences between configurations
\end{itemize}


%\subsection{Ereco, yreco}
%An additional study to be commenced is the mapping from Ereco \& yreco into other kinematic variables (i.e. $Q^2$, $q_0$ vs. $q_3$), and the ability to similarly map model variations.


\begin{thebibliography}{1}

\bibitem{GENIE}
Andreopoulos, C. \textit{et al}.
\textit{The GENIE Neutrino Monte Carlo Generator}.
Nucl.Instrum.Meth. A614 (2010) 87-104 arXiv:0905.2517 [hep-ph] FERMILAB-PUB-09-418-CD

\bibitem{NEUT}
Hayato, Yoshinari 
\textit{A neutrino interaction simulation program library NEUT}.
Acta Phys.Polon. B40 (2009) 2477-2489

\bibitem{NUWRO}
T. Golan, J.T. Sobczyk, J. Zmuda
\textit{NuWro: the Wrocław Monte Carlo Generator of Neutrino Interactions}.
Nuclear Physics B - Proceedings Supplements 229 (2012) 499

\bibitem{NUISANCE}
P. Stowell, C. Wret, C. Wilkinson, L. Pickering, S. Cartwright, Y. Hayato, K. Mahn, K.S. McFarland, J. Sobczyk, R. Terri, L. Thompson, M.O. Wascko, Y. Uchida
\textit{NUISANCE: a neutrino cross-section generator tuning and comparison framework}
arXiv:1612:07393v2

\end{thebibliography}


\end{document}




