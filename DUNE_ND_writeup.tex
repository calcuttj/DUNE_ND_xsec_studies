\setlength{\headheight}{15pt}
\documentclass[12pt]{article}
\usepackage{fancyhdr}
\lhead{}
\chead{}
\rhead{}
\renewcommand{\headrulewidth}{0pt}
\pagestyle{fancy}
\usepackage{graphicx}
\usepackage[top=2cm,bottom=3cm]{geometry}
\usepackage[svgnames]{xcolor}
\usepackage[colorlinks=true,linkcolor=DarkBlue,citecolor=DarkBlue]{hyperref}
\usepackage{xspace}
\usepackage{rotating}
\usepackage{units}
%\usepackage{subfig}
%\usepackage{amssymb, amsmath}
\usepackage{amsmath}
\usepackage{authblk}
\usepackage{lineno}
\usepackage{listings} 
\usepackage[normalem]{ulem}
%\usepackage{placeins}
\usepackage[section]{placeins}

\usepackage{SIunits}
\usepackage{hepunits}
\usepackage{hepparticles}
\usepackage{cancel}
\usepackage{hepnames}
\usepackage{epstopdf}
\usepackage{mathtools}
\usepackage{caption}
\usepackage[aboveskip=-10pt]{subcaption}
\usepackage[capitalise]{cleveref}
\usepackage{braket}
\usepackage{slashed}
\usepackage{subfiles}

\newcommand{\todo}[1]{{\color{red} TODO: #1}}
\newcommand\red[1]{{\color{red}#1}}
\newcommand{\ccpi}{CC1$\pi^0$\xspace}
\newcommand{\ccpis}{CC$\pi^0$\xspace}
\newcommand{\ccpip}{CC1$\pi^+$\xspace}
\newcommand{\ncpi}{NC1$\pi^0$\xspace}
\newcommand{\ccqe}{CCQE\xspace}
\newcommand{\mares}{\ensuremath{M_A^\mathrm{res}}\xspace}
\newcommand{\ppi}{\ensuremath{|\mathbf{p}_{\pi^0}|}\xspace}
\newcommand{\mb}{MiniBooNE\xspace}
\newcommand{\minerva}{MINER\ensuremath{\nu}A\xspace}
\newcommand{\neut}{\textsc{neut}\xspace}
\newcommand{\nuance}{\textsc{nuance}\xspace}
\newcommand{\tmu}{\ensuremath{T_{\mu}}\xspace}
\newcommand{\pmu}{\ensuremath{|\textbf{p}_{\mu}|}\xspace}
\newcommand{\cost}{\ensuremath{\cos{\theta_{\mu}}}\xspace}
\newcommand{\enu}{\ensuremath{E_{\nu}}\xspace}
\newcommand{\qq}{\ensuremath{Q^{2}}\xspace}
\newcommand{\qqqe}{\ensuremath{Q^{2}_{\textrm{QE}}}\xspace}
\newcommand{\pf}{\ensuremath{p_{F}}\xspace}
\newcommand{\eb}{\ensuremath{E_{b}}\xspace}
\newcommand{\carb}{C\ensuremath{^{12}}\xspace}
\newcommand{\oxy}{O\ensuremath{^{16}}\xspace}
\newcommand{\ie}{i.e.\xspace}
\newcommand{\eg}{e.g.\xspace}
\newcommand{\ma}{\ensuremath{M_{\textrm{A}}}\xspace}
\newcommand{\maqe}{\ensuremath{M_{\textrm{A}}^{\textrm{QE}}}\xspace}
\newcommand{\numu}{\Pnum}
\newcommand{\nue}{\Pnue}
\newcommand{\numubar}{\APnum}
\newcommand{\nuebar}{\APnue}
\newcommand{\enuqerfg}{\ensuremath{E^{\nu}_{\textrm{QE,RFG}}}\xspace}
\newcommand{\enuqe}{\ensuremath{E^{\nu}_{\textrm{QE}}}\xspace}
\newcommand{\chisq}{\ensuremath{\chi^{2}}\xspace}
\newcommand{\chisqmin}{\ensuremath{\chi^{2}_{\textrm{min}}}\xspace}
\newcommand{\chtwo}{CH\ensuremath{_{2}}\xspace}
\newcommand{\wroclaw}{Wroc{\l}aw\xspace}
\newcommand{\km}{\kilo\meter\xspace}
\newcommand{\m}{\meter\xspace}
\newcommand{\evsq}{\eV\ensuremath{^{2}}\xspace}
\newcommand{\POD}{P{\O}D\xspace}
\newcommand{\ecal}{ECal\xspace}
\newcommand{\ecals}{ECals\xspace}
\newcommand{\dsecal}{Ds-ECal\xspace}
\newcommand{\vol}[4]{\ensuremath{#1\times#2\times\unit{#3}{#4}}\xspace}
\newcommand{\area}[3]{\ensuremath{#1\times\unit{#2}{#3}}\xspace}
\newcommand{\pizero}{\pi^{0}\xspace}
\newcommand{\kg}{\kilo\gram\xspace}
\newcommand{\lep}{\ell}
\newcommand{\mnn}{multi-nucleon--neutrino\xspace}
\newcommand{\elt}{\ensuremath{E_{<}}\xspace}
\newcommand{\egt}{\ensuremath{E_{>}}\xspace}


\renewcommand\Im{\operatorname{Im}}

\graphicspath{{figures/}}

\newif\ifpdf
\ifx\pdfoutput\undefined
   \pdffalse
\else
   \pdfoutput=1
   \pdftrue
\fi
\ifpdf
   \usepackage{graphicx}
   \usepackage{epstopdf}
   %\DeclareGraphicsRule{.eps}{pdf}{.pdf}{`epstopdf #1}
   \pdfcompresslevel=9
\else
   \usepackage{graphicx}
\fi

\graphicspath{{figs/}}

\title{Current Status and Future Progess of DUNE ND studies}

\date{}
\begin{document}


\author[1]{Jake Calcutt}
\author[1]{Joshua Hignight}
\author[1]{Kendall Mahn}
\affil[1]{Michigan State University}
%\author[2]{Joshua Hignight}
%\author[3]{Kendall Mahn}


\maketitle
\thispagestyle{fancy}
%\linenumbers
%\begin{abstract}
%This report is a review of the current implementation of cross section related sources of systematic uncertainty for the DUNE Near Detector Taskforce (ND TF), charged with evaluating three possible near detector configurations. It identifies critical sources of systematic uncertainty, some of which are already covered, suggests studies to further improve the current uncertainty implementation, and summarizes future improvements to the systematic uncertainty outside the scope of the ND TF. 
%\end{abstract}
%A clear statement of what the committee deems to be an ideal (practical) scenario would be an excellent start. We can bring that to VALOR to determine what they can implement, and then ask the committee for some feedback on the limitations of the final VALOR parameterization to put in the report along with their full recommendations. The full simulation and analysis chain should not die with the task force, and the final report should include recommendations for what should be studied beyond the TF timeline.

\section{Overview}\label{sec:view}

%This document serves as a writeup detailing the current status of the DUNE ND studies and the progress that has been made so far, as well as a plan on how to further the studies in the near future. 

The Deep Underground Neutrino Experiment (DUNE) is a next-generation Long Baseline neutrino experiment designed to search for CP violation and establish the mass hierarchy. It consists of both a Near and Far Detector separated by 1300km and uses a neutrino beam created at Fermilab.\cite{DUNE_CDR1}. While the design of the Liquid Argon Far Detector has been finalized, there is still ongoing effort in deciding the configuration of the Near Detector. The main near detector design includes a Fine-Grained Tracker (FGT) with possible inclusion of additional detectors. Under consideration are a Liquid (LArTPC) or High Pressure Gaseous Argon TPC (GArTPC). DUNE's physics goals require systematic uncertainties in the interaction model to be below the 2\% limit after a near-to-far extrapolation\cite{DUNE_review}. The focus of this document is to quantify how the neutrino interaction model could affect DUNE's physics goals. This work was done concurrent with the DUNE ND Taskforce (TF), and so the studies investigate possible weaknesses and limitations of the current parameterization of neutrino interaction uncertainties. The studies also attempt to quantify how the three near detector configurations are individually sensitive to the neutrino interaction model.

Below is a list of the studies we have conducted for this work.
\begin{itemize}
\item The different abilities of the 3 ND configurations to reconstruct the neutrino energy, and how these compare to the Far Detector.
	\begin{itemize}
	\item Difference from true neutrino energy.
	\item Neutron multiplicities and neutron energy - Essentially a baseline of how much energy is not reconstructed.
	\item Proton \& pion multiplicity and momentum.
	\item Nucleon multiplicity vs. invariant hadronic mass.
	\end{itemize}
\item Choice of parameterization
	\begin{itemize}
	\item The sufficiency of a pure-$Q^2$ parameterization - are variations in models canceled out in a near to far extrapolation.
	\item Does the $Q^2$ parameterization ignore uncertainties present in $q_0 \textrm{ vs. } q_3$ phase space?
	\end{itemize}

\end{itemize}

%The focus of the work described in this document is then to quantify the abilities of the standalone FGT detector and additional LAr/GAr TPC to achieve this limit in the near-to-far extrapolation. 
%\begin{enumerate}
%\item Review the available material describing the handling of cross-section uncertainties within the DUNE Near Detector Task Force in the (\url{http://docs.dunescience.org:8080/cgi-bin/ShowDocument?docid=1291}{VALOR TN} sections 3 and 4 and the material presented at the 10am CT Thu July 14 ND Physics Working Group Meeting).
%\item Recommend changes within the existing framework that would better describe the current level of uncertainty in neutrino-nucleus interaction physics. These recommendations should be prioritized w.r.t. how crucial they are to the extraction of oscillation parameters (especially CP violation) and should take into account the limited timescale and manpower of the the NDTF (initial report Sept 2016, final report March 2017 ? about 0.5 FTE of cross-section resources).
%\item Make suggestions for studies that would help confirm the assumptions and/or results of the NDTF, should the time and manpower for these studies become available.
%\item  Make recommendations for a long term strategy for DUNE to study ND capabilities to i) measure neutrino-nucleus interactions and ii) constrain oscillation physics systematic uncertainties,  beyond the scope of the task force (task force charge can be found at this \url{https://docs.google.com/presentation/d16yY5CzRwo_243jpBzhVVeunvxYjIb7FzoS745RwURgw/edit#slide=id.p7}{link}.
%\end{enumerate}

\section{Methods}
%As the analysis techniques for DUNE are developed, checks on the cross section model are necessary. Variations arise in the different handling of Final State Interaction (FSI) effects by Monte Carlo event generators, as well as choice of Near Detector configuration. Sets of neutrino and antineutrino events on Argon-40 are produced with Monte Carlo Generators - GENIE\cite{GENIE} version 2.10, using an RFG model, NEUT\cite{NEUT} version 5.3.6, using Nieves et. al RPA+2p2h/MEC $M_A$ = 1.01, and NuWro\cite{NUWRO} version 11, using LFG + RPA + Nieves et. al - according to the 2015 DUNE CDR fluxes. The $\nu_\mu$ Near Detector and Far Detector fluxes used in this work are showin in Figure~\ref{fig:dune_flux}.  The various data sets are passed through the NUISANCE\cite{NUISANCE} software to reduce the various outputs to a common format, in turn saving all final state particle information for each event. 
Neutrino interaction modelling is of great interest to current and future oscillation experiments. Currently, there are a few different neutrino interaction software packages, so called "generators". While it is not guaranteed that the physics of nature corresponds to any of these software packages, the different choices made by the generators can act as a proxy for possible variations in the neutrino interaction model. The models considered span a range of different final state interaction models and QE, 2p2h models. These generators are GENIE\cite{GENIE} version 2.10, using an RFG model, NEUT\cite{NEUT} version 5.3.6, using Nieves et. al RPA+2p2h/MEC $M_A$ = 1.01, and NuWro\cite{NUWRO} version 11, using LFG + RPA + Nieves et. al. 
	
Sets of $\nu$ and $\overline{\nu}$ events with an Argon-40 target are produced with these generators according to the 2015 DUNE CDR fluxes, shown in Figure~\ref{fig:dune_flux}. (NEED TO FIND THE OSCILLATION PARAMETERS) The various data sets are then passed through the NUISANCE\cite{NUISANCE} software to reduce the outputs to a common format, in turn saving all final state particle information for each event. 

	All studies shown look at the effect of differences between the near and the far detector fluxes. However, we also include the different estimated efficiencies of each of the three near detector configurations. Efficiency information is included with the following procedure. Each particle is then randomly accepted or rejected by throwing a random number and checking against the efficiency according to the particle's momentum.  At time of this writing, only a robust description of the FGT efficiency is available, but other configurations will be added to subsequent versions of this note. An example of the efficiency for protons in the FGT is given in Figure~\ref{fig:FGT_proton_effs}. Simple thresholds for protons are applied for the GAr - 100 MeV/c - and LAr ND and FD - 200 MeV/c. Currently, we do not have the efficiencies for $\mu$, $\pi^{+,-}$, and $\pi^{0}$ in the LAr and GAr, and so the detectors are assumed to be perfectly efficient (no rejections) to these particles. $\pi^0$ efficiency information is also currently missing in the FGT configuration, and so none are rejected. 
\begin{figure}[h]
\centering
\minipage{.5\textwidth}
\includegraphics[width=\linewidth]{Dune_Flux/numu_ND_flux.png}
\endminipage
\minipage{.5\textwidth}
\includegraphics[width=\linewidth]{Dune_Flux/numu_FD_flux.png}
\endminipage
\caption{$\nu_\mu$ flux at DUNE ND (Left) and FD (Right)}
\label{fig:dune_flux}
\end{figure}
\FloatBarrier

\begin{figure}[h]
\centering
\minipage{.5\textwidth}
\includegraphics[width=\linewidth]{eff_plots/fgt_trkeff_proton.png}
\endminipage
\minipage{.5\textwidth}
\includegraphics[width=\linewidth]{eff_plots/fgt_pideff_proton.png}
\endminipage
\caption{Left: Tracking efficiency for protons in FGT. Right: PID efficiency for protons in FGT. Total efficiency for a given proton momentum is given by the product of the two efficiencies.}
\label{fig:FGT_proton_effs}
\end{figure}

%\newline

\section{Reconstructed Energy}\label{sec:Reco}
%\subsection{Final State, Reconstructed Energy}

A framework for investigating variations in reconstructed energy calculations between different generators and ND configurations has been developed. %However, because it is calculated only by summing the final state energy (generator truth information), this is largely incomplete, and it serves only as a starting point for more robust studies. The full studies will include detector efficiencies and kinematic cuts to fully display the differences between models and configurations. 
Currently, the estimate for the incident neutrino energy is calculated by summing the total energy from final state leptons and pions (all charges) and the kinetic energy of final state protons after passing efficiencies through the data sets. All neutrons are assumed undetectable. 
This can be summed up in Equation~\ref{eq:ereco}:
\begin{equation}
E_{reco} = \Sigma {E_{lep}} + \Sigma E_{\pi} + \Sigma (E_{prot} - M_{prot})
\label{eq:ereco}
\end{equation}
where $E_{lep}$ is the energy of the outgoing lepton, $E_{\pi}$ is the energy of the pion, and $E_{prot}$ and $M_{prot}$ are the energy and mass of the proton.
%This is achieved by using the NUISANCE software package to reduce the output from various Monte Carlo neutrino event generators (including GENIE, NEUT, NUWRO, as well as the nuclear reaction and transport simulation software, GiBuu) into a common format that can easily be analyzed. 
%Work has begun to include tracking/PID efficiencies to accept or reject final state particles. So far, only the full information (barring $\pi^0$ efficiencies) for the FGT is available, but the inclusion of the LAr or GAr TPCs will be easily implemented. 

%\subsection{$E_\nu - E_{reco}$} 
\subsection{Difference from True Neutrino Energy}
\label{subsec:EDiff}
To investigate these variations, the difference between true and reconstructed neutrino energy from each generator - where NuWro and NEUT distributions have been normalized to GENIE - are plotted for each near detector configuration as well as the far detector, and for different reaction types. The reaction types considered are true-CCQE, true-2p2h, CC0$\pi$, CC1$\pi$, and CCOther. True-CCQE and true-2p2h are both defined as MC-level CCQE/2p2h interaction with 1 reconstructed lepton. CC0$\pi$ is defined as 0 $\pi^{\pm}$, 1 lepton, and any number of protons and $\pi^0$ after reconstruction. CC1$\pi$ is defined as 1 $\pi^{\pm}$, 1 lepton, and any number of protons and $\pi^0$ after reconstruction. Finally, CCOther is defined as any final state with 1 lepton and any number of hadrons reconstructed.

In $\nu_\mu$ mode: for all reaction types except CCOther, all three generators seem to have similar differences in all detector configurations. This is evident in Figure~\ref{fig:numu_Etrue_ereco_FGT_CCQE_and_CCOther} - which includes CCQE and CCOther interactions in the ND with FGT efficiencies - and the figures in Appendix~\ref{app:Ereco_app}. For CCOther, all three generators appear to have a long tail extending to high discrepancy regions - well past that of CCQE for example - though NuWro has a higher amount of events in the $\Delta E$ = 0.5 GeV to 2 GeV region. 
\begin{figure}[h]
\centering
\minipage{.5\textwidth}
\includegraphics[width=\linewidth]{Ereco_Etrue/numu_FGT_CCQE.png}
\endminipage
\minipage{.5\textwidth}
\includegraphics[width=\linewidth]{Ereco_Etrue/numu_FGT_CCOther.png}
\endminipage
\caption{Difference between true and reconstructed Neutrino Energy in the FGT ND. CCQE events seem to have similar discrepancies, while CCOther mode has large tails in all three generators with NuWro having more events with higher discrepancy than the other two. }
\label{fig:numu_Etrue_ereco_FGT_CCQE_and_CCOther}
\end{figure}
\FloatBarrier

Meanwhile, for $\overline{\nu_\mu}$ CCQE, 2p2h, and CC0$\pi$ events NuWro consistently has $\Delta E$ closer to 0 than the other generators as shown in Figure~\ref{fig:numubar_Etrue_ereco_FGT_CCQE_2p2h_CC0Pi}. In CC1$\pi$ and CCOther NuWro again has more events where true neutrino energy is underestimated by reconstruction, shown in Figure~\ref{fig:numubar_Etrue_ereco_FGT_CC1Pi_CCOther}. Additionally, there exists a large difference in the distribution of events from NEUT and NuWro between the FGT and LAr FD. This is slightly better in the LAr ND - as expected due to our naive LAr efficiencies. This can be seen in Figure~\ref{fig:numubar_Etrue_ereco_CCOther_diffs}.
\begin{figure}[h]
\centering
\minipage{.3\textwidth}
\includegraphics[width=\linewidth]{Ereco_Etrue/numubar_FGT_CCQE.png}
\endminipage
\minipage{.3\textwidth}
\includegraphics[width=\linewidth]{Ereco_Etrue/numubar_FGT_2p2h.png}
\endminipage
\minipage{.3\textwidth}
\includegraphics[width=\linewidth]{Ereco_Etrue/numubar_FGT_CC0Pi.png}
\endminipage
\newline
\minipage{.3\textwidth}
\includegraphics[width=\linewidth]{Ereco_Etrue/numubar_FD_CCQE.png}
\endminipage
\minipage{.3\textwidth}
\includegraphics[width=\linewidth]{Ereco_Etrue/numubar_FD_2p2h.png}
\endminipage
\minipage{.3\textwidth}
\includegraphics[width=\linewidth]{Ereco_Etrue/numubar_FD_CC0Pi.png}
\endminipage
\caption{Difference between true and reconstructed $E_\nu$. Top: At Near Detector with FGT efficiencies. Bottom: Far Detector with LAr efficiencies. Left to Right: CCQE, 2p2h, CC0$\pi$. NuWro can be seen to have consistently less discrepancy in reconstructed energy than NEUT and GENIE in CCQE, 2p2h, and CC0$\pi$. }
\label{fig:numubar_Etrue_ereco_FGT_CCQE_2p2h_CC0Pi}
\end{figure}
\begin{figure}[h]
\centering
\minipage{.5\textwidth}
\includegraphics[width=\linewidth]{Ereco_Etrue/numubar_FGT_CC1Pi.png}
\endminipage
\minipage{.5\textwidth}
\includegraphics[width=\linewidth]{Ereco_Etrue/numubar_FGT_CCOther.png}
\endminipage
\caption{NuWro has higher discrepancy in reconstructed energy than NEUT and GENIE in both CC1$\pi$ and CCOther.}
\label{fig:numubar_Etrue_ereco_FGT_CC1Pi_CCOther}
\end{figure}
\begin{figure}[h]
\centering
\minipage{.3\textwidth}
\includegraphics[width=\linewidth]{Ereco_Etrue/numubar_FD_CCOther.png}
\endminipage
\minipage{.3\textwidth}
\includegraphics[width=\linewidth]{Ereco_Etrue/numubar_FGT_CCOther.png}
\endminipage
\minipage{.3\textwidth}
\includegraphics[width=\linewidth]{Ereco_Etrue/numubar_LAr_CCOther.png}
\endminipage
\caption{NuWro CCOther events have differences in shape between the Far Detector and LAr FD. The LAr ND matches more closely, though this is expected because of our current naive efficiency assumptions.}
\label{fig:numubar_Etrue_ereco_CCOther_diffs}
\end{figure}
\FloatBarrier

%As a first check in the ability to reconstruct neutrino energy, I began investigating the differences between the total final state energy and initial state energy. I've verified with Callum that the NUISANCE framework successfully treats the particle stacks from the various generators, and any differences come from how the different generators handle FSI. 



\subsection{Neutron Multiplicities \& Energy}
\label{sec:N_multiplicities_Energy}

Differences between models in the number of final state neutrons and the total energy into FS neutrons can largely affect reconstruction of neutrino energy. Large variations in reconstructed energy can arise due to missing energy caused by the inability to detect neutrons in the various models.  

To investigate this, we have looked at GENIE, NEUT, and NUWRO to see if the different models showed a large difference in the neutron energy and multiplicity.  
This was done for CCQE-like, CC1$\pi$, 2P2H, and everything else (``Other'') interactions and neutrinos as well as anti-neutrinos.  
In all cases, the generators agreed rather well with each other even though there were some differences in the neutron multiplicity. 
These differences only account for a small fraction of the events.
Figure~\ref{fig:Neutron_multi_2p2h_ND} shows an example of this for 2P2H neutrino events, while the other interaction modes can be found in Appendix~\ref{app:Neutron_Multiplicities}
  
\begin{figure}
\centering
\begin{subfigure}[b]{0.32\textwidth}
  \includegraphics[width=\textwidth]{nneutrons_v_total_ene/Nneutrons_Total_ENe_2p2h_GENIE_ND_numu.pdf}
\end{subfigure}
\begin{subfigure}[b]{0.32\textwidth}
  \includegraphics[width=\textwidth]{nneutrons_v_total_ene/Nneutrons_Total_ENe_2p2h_NEUT_ND_numu.pdf}
\end{subfigure}
\begin{subfigure}[b]{0.32\textwidth}
  \includegraphics[width=\textwidth]{nneutrons_v_total_ene/Nneutrons_Total_ENe_2p2h_NUWRO_ND_numu.pdf}
\end{subfigure}
\caption{The neutron multiplicity vs total neutron energy for 2P2H interactions for GENIE, NEUT, and NUWRO, respectively.  Even though they do show a different phase-space for the neutron multiplicity, they all agree where most of the energy lost to neutrons should be.  This is similar for other interaction types as well.}
\label{fig:Neutron_multi_2p2h_ND}
\end{figure}
Further more, the difference in the multiplicities between the generators becomes irrelevant after a ND to FD extraction, as can be seen in Figure~\ref{fig:Neutron_multi_2p2h_ND_FD}.
Here, the region where the most energy is lost agrees very well between the ND and FD for all generators and interaction types.
The areas with low statistics do show a disagreement, but very few events fall into this area.

\begin{figure}
\centering
\begin{subfigure}[b]{0.32\textwidth}
  \includegraphics[width=\textwidth]{nneutrons_v_total_ene/Nneutrons_Total_ENe_2p2h_GENIE_ND_FD_numu_norm.pdf}
\end{subfigure}
\begin{subfigure}[b]{0.32\textwidth}
  \includegraphics[width=\textwidth]{nneutrons_v_total_ene/Nneutrons_Total_ENe_2p2h_NEUT_ND_FD_numu_norm.pdf}
\end{subfigure}
\begin{subfigure}[b]{0.32\textwidth}
  \includegraphics[width=\textwidth]{nneutrons_v_total_ene/Nneutrons_Total_ENe_2p2h_NUWRO_ND_FD_numu_norm.pdf}
\end{subfigure}
\caption{The ratio of the ND to the FD for neutron multiplicity vs total neutron energy for 2P2H interactions for GENIE, NEUT, and NUWRO, respectively.  In the area where the largest amount of energy is lost to neutrons, low multiplicity and low energy, the agreement between the ND and FD is almost perfect.} 
\label{fig:Neutron_multi_2p2h_ND_FD}
\end{figure}
Care should still be taken when calculating the total neutrino of the event if only a calorimetric approach is used, as is the case in this document.
This is because as much as 50\% of the energy can be taken away by the neutron in CCQE-like events.
Even for 2P2H events, as shown in Figure~\ref{fig:Neutron_multi_ene_enu_2p2h_ND}, it can be as much as 30\%.
The other interaction modes and anti-neutrino events can also be found in Appendix~\ref{app:Neutron_Multiplicities} and have a smaller fraction.

\begin{figure}
\centering
\begin{subfigure}[b]{0.32\textwidth}
  \includegraphics[width=\textwidth]{nneutrons_ene_enu/Nneutrons_Enu_true_2p2h_GENIE_ND_numu_norm.pdf}
\end{subfigure}
\begin{subfigure}[b]{0.32\textwidth}
  \includegraphics[width=\textwidth]{nneutrons_ene_enu/Nneutrons_Enu_true_2p2h_NEUT_ND_numu_norm.pdf}
\end{subfigure}
\begin{subfigure}[b]{0.32\textwidth}
  \includegraphics[width=\textwidth]{nneutrons_ene_enu/Nneutrons_Enu_true_2p2h_NUWRO_ND_numu_norm.pdf}
\end{subfigure}
\caption{Neutron multiplicity vs total neutron energy divide by the neutrino energy for 2P2H interactions from GENIE, NEUT, and NUWRO, respectively.  For low multiplicity, the neutron carries away a significant fraction of the neutrino energy.} 
\label{fig:Neutron_multi_ene_enu_2p2h_ND}
\end{figure}
\FloatBarrier

 Additionally, investigations in the ability for errors in GENIE to cover the differences between models have been started. 

\subsection{Proton/Pion Multiplicity and Momentum}
\label{sec:Particle_multiplicities_momentum}

Nucleon multiplicity and momentum distributions offer similar information as do N vs. E distributions, while specifically looking at protons and charged pions along with detector thresholds can enlighten the ability of the detectors' reconstruction capabilities. For these studies, the 3-momentum of the final state (after efficiencies) protons or charged pions are summed. The magnitude is then plotted against the multiplicity for the specific particle type. 
\begin{figure}[h]
\minipage{.3\textwidth}
\includegraphics[width=\linewidth]{eff_N_P/FGT/protons/ratios/CCQE_NEUT_GENIE_numu_near_N_P.png}
\endminipage
\minipage{.3\textwidth}
\includegraphics[width=\linewidth]{eff_N_P/FGT/protons/ratios/CCQE_NEUT_GENIE_numu_far_N_P.png}
\endminipage
\minipage{.3\textwidth}
\includegraphics[width=\linewidth]{eff_N_P/FGT/protons/ratios/CCQE_NEUT_GENIE_numu_NF_N_P.png}
\endminipage
\caption{Nprotons vs. P protons distributed events using DUNE flux. Left: Ratio of NEUT to GENIE output at ND with FGT efficiencies, Mid: Ratio of NEUT to GENIE output at FD with LAr efficiencies, Right: Double ratio of NEUT to GENIE, Near to Far}
\end{figure}
\FloatBarrier
%\begin{equation}
%|\vec{P_{total,i}}| = |\Sigma \vec{P_i}|
%\end{equation}

\section{Parameterization}
%The VALOR group\cite{VALOR} has led multiple T2K oscillation analyses and has contributed to most published T2K oscillation papers(I TOOK THIS ALMOST WORD FOR WORD FROM THE VALOR DOC BUT IDK WHAT TO WRITE FOR IT. IF CITING IT IS NOT GOOD ENOUGH I NEED TO CHANGE THIS). It also has contributed to to optimisation studies for DUNE, this being a source of motivation for this work. 
The VALOR software package that has been used on multiple T2K oscillation analyses, and had recently been used to study the DUNE near detector configurations. The group uses MC templates to map between reconstructed and true information for various reactions. Currently, $Q^2$ parameterizations have been suggested by VALOR for various MC templates and are defined for both neutrinos and antineutrinos as: 
\begin{itemize}
\item CCQE with bins \{0 - 0.20, 0.20 - 0.55, $>$0.55\} $\textrm{GeV}^2$
\item CC1$\pi^{\pm}$ with bins \{0 - 0.30, 0.30 - 0.80, $>$0.80\} $\textrm{GeV}^2$
\item CC1$\pi^0$ with bins \{0 - 0.35, 0.35 - 0.90, $>$0.90\} $\textrm{GeV}^2$
\item CCOther with 1 $Q^2$ bin
\end{itemize}
The following portion of this work is devoted to investigating the sufficiency of these $Q^2$ parameterizations in treating CCQE and 2p2h model - and excluding FSI  - uncertainties in the various detector configurations. To do this, we look for evidence of $Q^2$ shape differences in CCQE and 2p2h events.

The following procedure describes how this is explored. Ratios of either NEUT or NuWro to GENIE - referred to as 'single ratios' and with NEUT and NuWro both normalized to GENIE - at the ND and FD offer insight into model variations and how these depend on ND configuration. 
The effect of a near to far extrapolation is approximated by taking a 'double ratio' between the near and far single ratios. To be explicit and to avoid confusion, these are defined in Equations ~\ref{eq:single_ratio} and ~\ref{eq:double_ratio}, where 'Other MC' refers to either NEUT or NuWro and 'Near' can be replaced by 'Far' in the single ratio. We note this is a crude approximation to give intuition about the problem for just a single reaction process; a fit to the ND spectrum will have multiple reactions in a given topology which can complicate the determination of physics effects for a single reaction.

\begin{equation}
\label{eq:single_ratio}
\textrm{ND Single Ratio} = \frac{\textrm{(Other MC @ Near Detector)}}{\textrm{(GENIE @ Near Detector)}}
\end{equation}

\begin{equation}
\label{eq:double_ratio}
\textrm{Double Ratio} = \frac{\textrm{(ND Single Ratio)}}{\textrm{(FD Single Ratio)}}
\end{equation}



\subsection{$Q^2$ studies}
\label{subsec:q2}
%An obvious first step in the investigation of $Q^2$ parameterization sufficiency is to consider the relative change between the models and this change as it is extrapolated between the ND and FD Detector. 
The single and double ratio procedure described above is conducted using events created by all three models at the ND and FD and distributed according to $Q^2$. This is done separately with and without efficiencies, for both $\nu_{\mu}$ and $\overline{\nu_{\mu}}$, and for true-CCQE and true-2p2h.
%, true-2p2h, CC0$\pi^{\pm}$, CC1$\pi^{\pm}$, and CCOther as defined in Section~\ref{subsec:EDiff}

For $\nu_{\mu}$ CCQE events, both NEUT and NuWro differ significantly from GENIE in $Q^2$ distributions, as seen in the NEUT to GENIE and NuWro to GENIE single ratios. Despite this, the double ratio distributions remain relatively flat and close to 1 for distributions without efficiencies, shown in Figure~\ref{fig:Q2_ccqe_no_eff}. For double ratios with efficiencies applied, though variations are greater in the higher end of these distributions, they do appear centered close to or near 1 as seen in Figure~\ref{fig:Q2_ccqe_FGT_eff}. From this, it appears that the VALOR $Q^2$ binning is suitable to handle variations.
\begin{figure}[h]
\minipage{.5\textwidth}
\includegraphics[width=\linewidth]{Q2/nominal/ratios/CCQE_NEUT_GENIE_numu_near_Q2.png}
\endminipage
\minipage{.5\textwidth}
\includegraphics[width=\linewidth]{Q2/nominal/ratios/CCQE_NEUT_GENIE_numu_NF_Q2.png}
\endminipage
\caption{$Q^2$ distributed $\nu_{\mu}$ events using DUNE flux, no efficiencies applied. Left: Ratio of NEUT to GENIE output at ND, Right: Double ratio of NEUT to GENIE, Near to Far. Of note is the relative flatness throughout the double ratio compared to the single ratio at the ND. Note that the VALOR binning is separated by lower bin edges of (0, 0.20, and 0.55) $\textrm{GeV}^2$}
\label{fig:Q2_ccqe_no_eff}
\end{figure}
\begin{figure}[h]
\minipage{.5\textwidth}
\includegraphics[width=\linewidth]{eff_Q2/FGT/ratios/CCQE_NEUT_GENIE_numu_near_Q2.png}
\endminipage
\minipage{.5\textwidth}
\includegraphics[width=\linewidth]{eff_Q2/FGT/ratios/CCQE_NEUT_GENIE_numu_NF_Q2.png}
\endminipage
\caption{$Q^2$ distributed $\nu_{\mu}$ events using DUNE flux, FGT efficiencies applied to ND and LAr effiencies applied to FD. Left: Ratio of NEUT to GENIE output at ND (high $Q^2$ region out of bounds of plot), Right: Double ratio of NEUT to GENIE, Near to Far. More variations do arise in the higher regions of the double ratio, but still appear centered close to or around 1. Note that the VALOR binning is separated by lower bin edges of (0, 0.20, and 0.55) $\textrm{GeV}^2$}
\label{fig:Q2_ccqe_FGT_eff}
\end{figure}
\FloatBarrier

However, for $\overline{\nu_{\mu}}$ CCQE events, we observe distortions in the $Q^2$ distribution even in the double ratios. This is true without efficiencies as well as all three configurations, and for both NEUT to GENIE and NuWro to GENIE ratios. It is particularly worse in NuWro. This is displayed in Figure~\ref{fig:Q2_ccqe_bar}. Note that the last bin is not visible in the last NuWro plot as it has been cropped out, Appendix~\ref{app:Q2_app} has the full plot. The double ratios are no longer centered close to one, and so the VALOR binning seems insufficient for $\overline{\nu_{\mu}}$ CCQE. %This trend of larger variations in $\overline{\nu_{\mu}}$ as opposed to $\nu_{\mu}$ events continues in all reaction types except for 2p2h and can be seen in Appendix~\ref{app:Q2_app} where all plots are stored.
\begin{figure}[h]
\minipage{.5\textwidth}
\includegraphics[width=\linewidth]{Q2/nominal/ratios/CCQE_NEUT_GENIE_numubar_NF_Q2.png}
\endminipage
\minipage{.5\textwidth}
\includegraphics[width=\linewidth]{Q2/nominal/ratios/CCQE_NuWro_GENIE_numubar_NF_Q2.png}
\endminipage
\newline
\minipage{.5\textwidth}
\includegraphics[width=\linewidth]{eff_Q2/FGT/ratios/CCQE_NEUT_GENIE_numubar_NF_Q2.png}
\endminipage
\minipage{.5\textwidth}
\includegraphics[width=\linewidth]{eff_Q2/FGT/ratios/CCQE_NuWro_GENIE_numubar_NF_Q2_zoomed.png}
\endminipage
\caption{$Q^2$ $\overline{\nu_{\mu}}$ CCQE events using DUNE flux. Top: No efficiences, Bottom: FGT efficiencies applied to ND, LAr efficiencies applied to FD. Left to right: NEUT to GENIE double ratio, NuWro to GENIE double ratio. Large amounts of variations throughout all distributions, particularly in NuWro, showing the double ratios are no longer centered close to or around 1. Note: The last bin in the botom right plot has been cropped out. Note that the VALOR binning is separated by lower bin edges of (0, 0.20, and 0.55) $\textrm{GeV}^2$.}
\label{fig:Q2_ccqe_bar}
\end{figure}
\FloatBarrier

For 2p2h $\nu_{\mu}$ events, the double ratios from both NEUT and NuWro remain centered close to or around 1 in the low $Q^2$ region, similar to in CCQE, with and without efficiencies applied. The upper $Q^2$ regions do experience some deviation from 1, as seen in Figure~\ref{fig:Q2_2p2h_numu} However, as these are only in the upper $Q^2$ region, there are much fewer events. So, as opposed to the CCQE $\overline{\nu_{\mu}}$ distributions where the distortions arose throughout, it is conluded that the VALOR binning still holds, similar to CCQE $\nu_{\mu}$.
\begin{figure}[h]
\minipage{.5\textwidth}
\includegraphics[width=\linewidth]{Q2/nominal/ratios/2p2h_NEUT_GENIE_numu_NF_Q2.png}
\endminipage
\minipage{.5\textwidth}
\includegraphics[width=\linewidth]{Q2/nominal/ratios/2p2h_NuWro_GENIE_numu_NF_Q2.png}
\endminipage
\newline
\minipage{.5\textwidth}
\includegraphics[width=\linewidth]{eff_Q2/FGT/ratios/2p2h_NEUT_GENIE_numu_NF_Q2.png}
\endminipage
\minipage{.5\textwidth}
\includegraphics[width=\linewidth]{eff_Q2/FGT/ratios/2p2h_NuWro_GENIE_numu_NF_Q2.png}
\endminipage
\caption{$Q^2$ $\overline{\nu_{\mu}}$ 2p2h events using DUNE flux. Top: No efficiences, Bottom: FGT efficiencies applied to ND, LAr efficiencies applied to FD. Left to right: NEUT to GENIE double ratio, NuWro to GENIE double ratio. Large amounts of variations are present resulting from low statistics in the higher end of all distributions, but the lower end holds close to 1. Note that the VALOR binning is separated by lower bin edges of (0, 0.20, and 0.55) $\textrm{GeV}^2$.}
\label{fig:Q2_2p2h_numu}
\end{figure}
\FloatBarrier

For $\overline{\nu_{\mu}}$ 2p2h events, the double ratios are relatively flat and close to 1 again in the low $Q^2$ regio. Adding efficiencies distorts this away from 1 throughout the distributions, even in the lower regions. See Figure~\ref{fig:Q2_2p2h} for this with FGT ND and LAr FD efficiencies applied. Note that bins 6 and 10 have been cropped out of the lower right plot, and the full plots are availabile in Appendix~\ref{app:Q2_app}. Similarly to $\overline{\nu_{\mu}}$ CCQE, this is evident that the VALOR binning is not sufficient for $\overline{\nu_{\mu}}$. 
\begin{figure}[h]
\minipage{.5\textwidth}
\includegraphics[width=\linewidth]{Q2/nominal/ratios/2p2h_NEUT_GENIE_numubar_NF_Q2.png}
\endminipage
\minipage{.5\textwidth}
\includegraphics[width=\linewidth]{Q2/nominal/ratios/2p2h_NuWro_GENIE_numubar_NF_Q2.png}
\endminipage
\newline
\minipage{.5\textwidth}
\includegraphics[width=\linewidth]{eff_Q2/FGT/ratios/2p2h_NEUT_GENIE_numubar_NF_Q2.png}
\endminipage
\minipage{.5\textwidth}
\includegraphics[width=\linewidth]{eff_Q2/FGT/ratios/2p2h_NuWro_GENIE_numubar_NF_Q2_zoomed.png}
\endminipage
\caption{$Q^2$ $\overline{\nu_{\mu}}$ 2p2h events using DUNE flux. Top: No efficiences, Bottom: FGT efficiencies applied to ND, LAr efficiencies applied to FD. Left to right: NEUT to GENIE double ratio, NuWro to GENIE double ratio. Large amounts of deviations from 1 throughout distributions with efficiencies applied. Note that the VALOR binning is separated by lower bin edges of (0, 0.20, and 0.55) $\textrm{GeV}^2$.}
\label{fig:Q2_2p2h}
\end{figure}
\FloatBarrier


\subsection{$q_0 \textrm{ vs. } q_3$ studies}
\label{subsec:q0q3}
In addition to studying the $Q^2$ ratios, the ratios in $q_0 \textrm{ vs. } q_3$ space were also investigated to explore possibilities that variations are highlighted when chaning the parameterization from $Q^2$ to $q_0 \textrm{ vs. } q_3$. It is possible that the $Q^2$ double ratios can appear flat, while $q_0 \textrm{ vs. } q_3$ distributed double ratios do not. If this is the case, a pure-$Q^2$ parameterization cannot sufficiently account for variations in model FSI effects. These studies were done in the exact same way as those in the previous section, only with the elements distributed in $q_0 \textrm{ vs. } q_3$ bins. Note: for the following distributions, the scale extends from 0 to 2. This was chosen for aesthetic reasons, and so any bins which appear to be at 2, could possibly be well above that number. This should not affect interpretations of the plots so much, since the point is to look at relative how close - the fainter blue and red or white bins - or far - the more solid blue and red bins - the ratios are from 1. 

For $\nu_{\mu}$ CCQE, the double ratios without efficiencies appear pretty flat for both NEUT and NuWro, as seen in Figure~\ref{fig:q0q3_numu_CCQE_no_eff}, though there is a bit of bin-to-bin variation in the upper right region. When adding efficiencies, the phase space is smeared out and  more bin-to-bin variation arises as shown in Figure~\ref{fig:q0q3_numu_CCQE_FGT_eff}.
%\newline
\begin{figure}[h]
\centering
\minipage{.5\textwidth}
\includegraphics[width=\linewidth]{q0_q3/nominal/ratios/CCQE_NEUT_GENIE_numu_NF_q3_q0.png}
\endminipage
\minipage{.5\textwidth}
\includegraphics[width=\linewidth]{q0_q3/nominal/ratios/CCQE_NuWro_GENIE_numu_NF_q3_q0.png}
\endminipage
\caption{$q_0 \textrm{ vs. } q_3$ $\nu_{\mu}$ events using DUNE flux without efficiencies applied. Left: Double ratio of NEUT to GENIE, Right: Double ratio of NuWro to GENIE. Relatively flat distributions, but with bin-to-bin variations}
\label{fig:q0q3_numu_CCQE_no_eff}
\end{figure}
\begin{figure}[h]
\centering
\minipage{.5\textwidth}
\includegraphics[width=\linewidth]{eff_q0_q3/FGT/ratios/CCQE_NEUT_GENIE_numu_NF_q3_q0.png}
\endminipage
\minipage{.5\textwidth}
\includegraphics[width=\linewidth]{eff_q0_q3/FGT/ratios/CCQE_NuWro_GENIE_numu_NF_q3_q0.png}
\endminipage
\caption{$q_0 \textrm{ vs. } q_3$ $\nu_{\mu}$ events using DUNE flux, FGT efficiencies applied to ND and LAr effiencies applied to FD. Left: Double ratio of NEUT to GENIE, Right: Double ratio of NuWro to GENIE. Distributions smeared out and variations between bins are higher than without efficiencies}
\label{fig:q0q3_numu_CCQE_FGT_eff}
\end{figure}
\FloatBarrier

For $\overline{\nu_{\mu}}$ CCQE, however, there is a dramatic difference between NuWro and NEUT when looking at the ratios without efficiencies. Shown in Figure~\ref{fig:q0q3_numubar_CCQE_no_eff}, it can be seen that the double ratio is consistently higher in the upper-right region of the NuWro/GENIE distribution as opposed to NEUT/GENIE, which is more similar to the corresponding $\nu_{\mu}$ ratio. When adding efficiencies, much of the upper right region is cut out in GENIE and NEUT, shown in Figure~\ref{fig:q0q3_numubar_CCQE_events_FGT_eff}, resulting in empty bins in both double ratios. Despite the distorted phase space, the NEUT/GENIE $\overline{\nu_{\mu}}$ CCQE double ratio appears to behave similarly to the corresponding $\nu_{\mu}$ CCQE double ratio, while NuWro has ratios greater than 1 throughout the shared phaes space, which is evident in Figure ~\ref{fig:q0q3_numubar_CCQE_FGT_eff}
\begin{figure}[h]
\centering
\minipage{.5\textwidth}
\includegraphics[width=\linewidth]{q0_q3/nominal/ratios/CCQE_NEUT_GENIE_numubar_NF_q3_q0.png}
\endminipage
\minipage{.5\textwidth}
\includegraphics[width=\linewidth]{q0_q3/nominal/ratios/CCQE_NuWro_GENIE_numubar_NF_q3_q0.png}
\endminipage
\caption{$q_0 \textrm{ vs. } q_3$ $\overline{\nu_{\mu}}$ events using DUNE flux without efficiencies applied. Left: Double ratio of NEUT to GENIE, Right: Double ratio of NuWro to GENIE. Relatively flat for NEUT/GENIE. NuWro/GENIE double ratio has heavy disagreement in upper-right region.}
\label{fig:q0q3_numubar_CCQE_no_eff}
\end{figure}
\begin{figure}[h]
\centering
\minipage{.3\textwidth}
\includegraphics[width=\linewidth]{eff_q0_q3/FGT/CCQE_RHC_ND_numubar_q3_q0_GENIE.png}
\endminipage
\minipage{.3\textwidth}
\includegraphics[width=\linewidth]{eff_q0_q3/FGT/CCQE_RHC_ND_numubar_q3_q0_NEUT.png}
\endminipage
\minipage{.3\textwidth}
\includegraphics[width=\linewidth]{eff_q0_q3/FGT/CCQE_RHC_ND_numubar_q3_q0_NuWro.png}
\endminipage
\caption{$q_0 \textrm{ vs. } q_3$ $\overline{\nu_{\mu}}$ events using DUNE flux, FGT efficiencies applied to ND and LAr effiencies applied to FD. Right to Left: GENIE events distribution, NEUT events distribution, NuWro events distribution. Both GENIE and NEUT cut out the upper right region of phase space as opposed to NuWro.}
\label{fig:q0q3_numubar_CCQE_events_FGT_eff}
\end{figure}
\begin{figure}[h]
\centering
\minipage{.5\textwidth}
\includegraphics[width=\linewidth]{eff_q0_q3/FGT/ratios/CCQE_NEUT_GENIE_numubar_NF_q3_q0.png}
\endminipage
\minipage{.5\textwidth}
\includegraphics[width=\linewidth]{eff_q0_q3/FGT/ratios/CCQE_NuWro_GENIE_numubar_NF_q3_q0.png}
\endminipage
\caption{$q_0 \textrm{ vs. } q_3$ $\overline{\nu_{\mu}}$ events using DUNE flux, FGT efficiencies applied to ND and LAr effiencies applied to FD. Left: Double ratio of NEUT to GENIE, Right: Double ratio of NuWro to GENIE. NEUT to GENIE double ratio appears similar to the $\nu_{\mu}$ double ratio, while NuWro to GENIE appears greater than 1 in shared bins.}
\label{fig:q0q3_numubar_CCQE_FGT_eff}
\end{figure}
\FloatBarrier
Each of the three generators have very different behavior for 2p2h $\nu_\mu$ and $\overline{\nu_{\mu}}$ events, as can be seen in the upper half of Figure~\ref{fig:q0q3_numu_2p2h_events}. This causes very restriced phase space in the double ratios. However, adding efficiencies smears out the phase space, causing more bins to be shared between generators, though there are still variations in the shapes. See the bottom half of Figure~\ref{fig:q0q3_numu_2p2h_events}. The double ratios seem to be mostly flat for both NEUT and NuWro in the FGT for $\nu_{\mu}$, seen in Figure~\ref{fig:q0q3_numu_2p2h_FGT_eff}. However for $\overline{\nu_{\mu}}$ 2p2h in the FGT, both NEUT to GENIE and NuWro to GENIE double ratios are generally greater than 1, seen in Figure~\ref{fig:q0q3_numubar_2p2h_FGT_eff}
\begin{figure}[h]
\centering
\minipage{.3\textwidth}
\includegraphics[width=\linewidth]{q0_q3/nominal/2p2h_FHC_ND_numu_q3_q0_GENIE.png}
\endminipage
\minipage{.3\textwidth}
\includegraphics[width=\linewidth]{q0_q3/nominal/2p2h_FHC_ND_numu_q3_q0_NEUT.png}
\endminipage
\minipage{.3\textwidth}
\includegraphics[width=\linewidth]{q0_q3/nominal/2p2h_FHC_ND_numu_q3_q0_NuWro.png}
\endminipage
\newline
\minipage{.3\textwidth}
\includegraphics[width=\linewidth]{eff_q0_q3/FGT/2p2h_FHC_ND_numu_q3_q0_GENIE.png}
\endminipage
\minipage{.3\textwidth}
\includegraphics[width=\linewidth]{eff_q0_q3/FGT/2p2h_FHC_ND_numu_q3_q0_NEUT.png}
\endminipage
\minipage{.3\textwidth}
\includegraphics[width=\linewidth]{eff_q0_q3/FGT/2p2h_FHC_ND_numu_q3_q0_NuWro.png}
\endminipage
\caption{$q_0 \textrm{ vs. } q_3$ $\nu_{\mu}$ events using DUNE flux, Top Row: No Efficiencies applied, Bottom Row: FGT efficiencies applied to ND and LAr effiencies applied to FD. Right to Left: GENIE events distribution, NEUT events distribution, NuWro events distribution. The phase spaces are very different without efficiencies, and restrict the coverage of the doubel ratios. With efficiencies, more phase space is shared.}
\label{fig:q0q3_numu_2p2h_events}
\end{figure}
\begin{figure}[h]
\centering
\minipage{.5\textwidth}
\includegraphics[width=\linewidth]{eff_q0_q3/FGT/ratios/2p2h_NEUT_GENIE_numu_NF_q3_q0.png}
\endminipage
\minipage{.5\textwidth}
\includegraphics[width=\linewidth]{eff_q0_q3/FGT/ratios/2p2h_NuWro_GENIE_numu_NF_q3_q0.png}
\endminipage
\caption{$q_0 \textrm{ vs. } q_3$ $\nu_{\mu}$ events using DUNE flux, FGT efficiencies applied to ND and LAr effiencies applied to FD. Left: Double ratio of NEUT to GENIE, Right: Double ratio of NuWro to GENIE. Ratios appear mostly flat.}
\label{fig:q0q3_numu_2p2h_FGT_eff}
\end{figure}
\begin{figure}[h]
\centering
\minipage{.5\textwidth}
\includegraphics[width=\linewidth]{eff_q0_q3/FGT/ratios/2p2h_NEUT_GENIE_numubar_NF_q3_q0.png}
\endminipage
\minipage{.5\textwidth}
\includegraphics[width=\linewidth]{eff_q0_q3/FGT/ratios/2p2h_NuWro_GENIE_numubar_NF_q3_q0.png}
\endminipage
\caption{$q_0 \textrm{ vs. } q_3$ $\overline{\nu_{\mu}}$ events using DUNE flux, FGT efficiencies applied to ND and LAr effiencies applied to FD. Left: Double ratio of NEUT to GENIE, Right: Double ratio of NuWro to GENIE. Both double ratios consistently higher than 1.}
\label{fig:q0q3_numubar_2p2h_FGT_eff}
\end{figure}
\FloatBarrier

For the rest of the reaction modes, the trend - that $\nu_{\mu}$ ratios appear to be relatively flat for both NEUT and NuWro, while $\overline{\nu_{\mu}}$ NuWro to GENIE double ratio is noticeably greater than 1 - continues. This can be seen in Appendix~\ref{app:q0q3_app}. Of note, however, is the shape of $\nu_\mu$ and $\overline{\nu_{\mu}}$ CCOther events before and after efficiencies are applied in each generator. As shown in Figure~\ref{fig:q0q3_numu_CCOther_events}, the left region of $q_0 \textrm{ vs. } q_3$ space is empty before efficiencies, but is filled in after efficiencies are applied. This corresponds to a low $Q^2$ region, which is shown to be filled in after efficiencies in the $Q^2$ distribution in Figure~\ref{fig:Q2_numu_CCOther_events}.
\begin{figure}[h]
\centering
\minipage{.3\textwidth}
\includegraphics[width=\linewidth]{q0_q3/nominal/CCOther_FHC_ND_numu_q3_q0_GENIE.png}
\endminipage
\minipage{.3\textwidth}
\includegraphics[width=\linewidth]{q0_q3/nominal/CCOther_FHC_ND_numu_q3_q0_NEUT.png}
\endminipage
\minipage{.3\textwidth}
\includegraphics[width=\linewidth]{q0_q3/nominal/CCOther_FHC_ND_numu_q3_q0_NuWro.png}
\endminipage
\newline
\minipage{.3\textwidth}
\includegraphics[width=\linewidth]{eff_q0_q3/FGT/CCOther_FHC_ND_numu_q3_q0_GENIE.png}
\endminipage
\minipage{.3\textwidth}
\includegraphics[width=\linewidth]{eff_q0_q3/FGT/CCOther_FHC_ND_numu_q3_q0_NEUT.png}
\endminipage
\minipage{.3\textwidth}
\includegraphics[width=\linewidth]{eff_q0_q3/FGT/CCOther_FHC_ND_numu_q3_q0_NuWro.png}
\endminipage
\caption{$q_0 \textrm{ vs. } q_3$ $\nu_{\mu}$ events using DUNE flux, Top Row: No Efficiencies applied, Bottom Row: FGT efficiencies applied to ND and LAr effiencies applied to FD. Right to Left: GENIE events distribution, NEUT events distribution, NuWro events distribution. The left $q_0 \textrm{ vs. } q_3$ $\nu_{\mu}$ region is being filled in after efficiencies are applied.}
\label{fig:q0q3_numu_CCOther_events}
\end{figure}
\begin{figure}[h]
\centering
\minipage{.3\textwidth}
\includegraphics[width=\linewidth]{Q2/nominal/CCOther_FHC_ND_numu_Q2_GENIE.png}
\endminipage
\minipage{.3\textwidth}
\includegraphics[width=\linewidth]{Q2/nominal/CCOther_FHC_ND_numu_Q2_NEUT.png}
\endminipage
\minipage{.3\textwidth}
\includegraphics[width=\linewidth]{Q2/nominal/CCOther_FHC_ND_numu_Q2_NuWro.png}
\endminipage
\newline
\minipage{.3\textwidth}
\includegraphics[width=\linewidth]{eff_Q2/FGT/CCOther_FHC_ND_numu_Q2_GENIE.png}
\endminipage
\minipage{.3\textwidth}
\includegraphics[width=\linewidth]{eff_Q2/FGT/CCOther_FHC_ND_numu_Q2_NEUT.png}
\endminipage
\minipage{.3\textwidth}
\includegraphics[width=\linewidth]{eff_Q2/FGT/CCOther_FHC_ND_numu_Q2_NuWro.png}
\endminipage
\caption{$Q^2$ events using DUNE flux, Top Row: No Efficiencies applied, Bottom Row: FGT efficiencies applied to ND and LAr effiencies applied to FD. Right to Left: GENIE events distribution, NEUT events distribution, NuWro events distribution. The low $Q^2$ region is being filled in after efficiencies are applied.}
\label{fig:Q2_numu_CCOther_events}
\end{figure}
\FloatBarrier

\subsection{Nucleon multiplicity vs. W}
\label{sec:Nucleon_multi_w}
Differences in mapping from $E_{reco}$ to true variables can arise from shape differences in nucleon multiplicities vs. W distributions. 
These distributions can also show where in the phase space most of the events shown in Section~\ref{sec:N_multiplicities_Energy} and Section~\ref{sec:Particle_multiplicities_Energy} exist.  A first look of this is shown in this section.

\subsubsection{Neutrons vs W}

Apart from the difference in neutron multiplicity discussed in Section~\ref{sec:N_multiplicities_Energy}, the exact phase space of these event in the neutron/W plane is slightly different.  
This is most pronounced for 2P2H events, as can be seen in Figure~\ref{Neutron_w_2p2h_ND}. 
The width of NEUT's peak is broader then GENIE's or NUWRO's while NUWRO allows for much higher W's.

\begin{figure}
\centering
\begin{subfigure}[b]{0.32\textwidth}
  \includegraphics[width=\textwidth]{nneutrons_w/Nneutrons_W_nuc_rest_2p2h_GENIE_ND_numu.pdf}
\end{subfigure}
\begin{subfigure}[b]{0.32\textwidth}
  \includegraphics[width=\textwidth]{nneutrons_w/Nneutrons_W_nuc_rest_2p2h_NEUT_ND_numu.pdf}
\end{subfigure}
\begin{subfigure}[b]{0.32\textwidth}
  \includegraphics[width=\textwidth]{nneutrons_w/Nneutrons_W_nuc_rest_2p2h_NUWRO_ND_numu.pdf}
\end{subfigure}
\caption{The neutron multiplicity vs W for 2P2H interactions from GENIE, NEUT, and NUWRO, respectively.  It can be seen that all three event generators have the peak of the distribution at the same place, but NEUT's peak is much broader while NUWRO allows for much higher W's.}
\label{fig:Neutron_w_2p2h_ND}
\end{figure}

However, Figure~\ref{fig:Neutron_w_2p2h_ND_FD} shows that the differences are not so important if a near to far extrapolation is used, as both the ND and FD have simular responses.
To see if the different models could have a larger affect on the physics results, we have taken a double ratio of the ND/FD and the different generators to GENIE.  
Interestingly, Figure~\ref{fig:Neutron_w_2p2h_ND_FD_GENIE} indicates that there would not be large affect if a few perecent difference between the models is acceptable.

\begin{figure}
\centering
\begin{subfigure}[b]{0.32\textwidth}
  \includegraphics[width=\textwidth]{nneutrons_w/Nneutrons_W_nuc_rest_2p2h_GENIE_ND_FD_numu_norm.pdf}
\end{subfigure}
\begin{subfigure}[b]{0.32\textwidth}
  \includegraphics[width=\textwidth]{nneutrons_w/Nneutrons_W_nuc_rest_2p2h_NEUT_ND_FD_numu_norm.pdf}
\end{subfigure}
\begin{subfigure}[b]{0.32\textwidth}
  \includegraphics[width=\textwidth]{nneutrons_w/Nneutrons_W_nuc_rest_2p2h_NUWRO_ND_FD_numu_norm.pdf}
\end{subfigure}
\caption{The ND/FD ratio of neutron multiplicity vs W for 2P2H interactions from GENIE, NEUT, and NUWRO, respectively.  The effects of the different phase space seen in Figure~\ref{fig:Neutron_w_2p2h_ND} is not great if a ND ro FD extropolation is used.  This can be seen by the ratio being very close to one at the peak of Figure~\ref{fig:Neutron_w_2p2h_ND} distribution.}
\label{fig:Neutron_w_2p2h_ND_FD}
\end{figure}

These results are also true for protons and pions, though pions have a much lower multiplicity.  
Those results, along with other interactions can be found in Appendix~\ref{app:Nucleon_W}.

\begin{figure}
\centering
\begin{subfigure}[b]{0.32\textwidth}
  \includegraphics[width=\textwidth]{nneutrons_w/Nneutrons_W_nuc_rest_2p2h_GENIE_NEUT_ND_FD_numu_norm.pdf}
\end{subfigure}
\begin{subfigure}[b]{0.32\textwidth}
  \includegraphics[width=\textwidth]{nneutrons_w/Nneutrons_W_nuc_rest_2p2h_GENIE_NUWRO_ND_FD_numu_norm.pdf}
\end{subfigure}
\caption{The double ratio of ND/FD and the event generators to GENIE of the neutron multiplicity vs W for 2P2H interactions.  Despite the differences seen in the W distributions, the double ratio shows suprisingly good agreement between the generators, indicating that the different models should not have a large impact on the final results.}
\label{fig:Neutron_w_2p2h_ND_FD_GENIE}
\end{figure}

\subsubsection{Protons vs W}

In the last section we saw a differnce in the phase space between the different generators.  
Now we will show an instance where the phase space is almost the same but there is a large difference between the ND and FD.  
Though here we are only showing the results for CC1$\pi$ via protons, the results are the same for CC-Other (anything not CCQE, CC1$\pi$, or 2P2H), and neutrons along with pions.
Figure~\ref{fig:Proton_w_res_ND} shows that GENIE, NEUT, and NUWRO have very similar distributions.

\begin{figure}
\centering
\begin{subfigure}[b]{0.32\textwidth}
  \includegraphics[width=\textwidth]{nprotons_w/Nprotons_W_nuc_rest_res_GENIE_ND_numu.pdf}
\end{subfigure}
\begin{subfigure}[b]{0.32\textwidth}
  \includegraphics[width=\textwidth]{nprotons_w/Nprotons_W_nuc_rest_res_NEUT_ND_numu.pdf}
\end{subfigure}
\begin{subfigure}[b]{0.32\textwidth}
  \includegraphics[width=\textwidth]{nprotons_w/Nprotons_W_nuc_rest_res_NUWRO_ND_numu.pdf}
\end{subfigure}
\caption{The proton multiplicity vs W for CC1$\pi$ interactions from GENIE, NEUT, and NUWRO, respectively.  It can be seen that all three event generators have the peak of their distribution at the same place and are almost identical}
\label{fig:Proton_w_res_ND}
\end{figure}

If a near to far detector extrapolation is used, however, there are much larger differences in region of most interest. 
Figure~\ref{fig:Proton_w_res_ND_FD} shows that there is a difference between the two detectors between 20\% and 40\% at the peak of the distribution.  
There is also a clear difference in lower W values compared to higher W values. 


\begin{figure}
\centering
\begin{subfigure}[b]{0.32\textwidth}
  \includegraphics[width=\textwidth]{nprotons_w/Nprotons_W_nuc_rest_res_GENIE_ND_FD_numu_norm.pdf}
\end{subfigure}
\begin{subfigure}[b]{0.32\textwidth}
  \includegraphics[width=\textwidth]{nprotons_w/Nprotons_W_nuc_rest_res_NEUT_ND_FD_numu_norm.pdf}
\end{subfigure}
\begin{subfigure}[b]{0.32\textwidth}
  \includegraphics[width=\textwidth]{nprotons_w/Nprotons_W_nuc_rest_res_NUWRO_ND_FD_numu_norm.pdf}
\end{subfigure}
\caption{The ND/FD ratio of the proton multiplicity vs W for CC1$\pi$ interactions from GENIE, NEUT, and NUWRO, respectively.  Unlike with 2P2H events shown for neutrons, there are clear differences between the ND and FD.  At the peak of the distribution this is between 20\% and 40\%.  Furthermore, there is a clear difference between W values below 2000 and ones above it.}
\label{fig:Proton_w_res_ND_FD}
\end{figure}

We can again see if the different models could have a larger affect on the physics results, by taking the double ratio as we did before.
Simular to before, Figure~\ref{fig:Proton_w_res_ND_FD_GENIE} indicates that there would not be large affect if a few perecent difference between the models is acceptable.
In the region of interest, the difference between GENIE and NEUT is negligable while it is small compared to NUWRO.


\begin{figure}
\centering
\begin{subfigure}[b]{0.32\textwidth}
  \includegraphics[width=\textwidth]{nprotons_w/Nprotons_W_nuc_rest_res_GENIE_NEUT_ND_FD_numu_norm.pdf}
\end{subfigure}
\begin{subfigure}[b]{0.32\textwidth}
  \includegraphics[width=\textwidth]{nprotons_w/Nprotons_W_nuc_rest_res_GENIE_NUWRO_ND_FD_numu_norm.pdf}
\end{subfigure}
\caption{The double ratio of ND/FD and the event generators to GENIE of the proton multiplicity vs W for CC1$\pi$ interactions.  Despite the differences seen in the single ratio, the double ratio shows suprisingly good agreement between the generators}
\label{fig:Proton_w_res_ND_FD_GENIE}
\end{figure}

It appears as though the pure-$Q^2$ paramterization could cover model variations for $\nu_{\mu}$ events. The near-to-far extrapolation plots - the double ratios - appear relatively flat in all reaction modes. However, the $\overline{\nu_{\mu}}$ events show that model variations are not covered in this extrapolation, at least when looking into the NuWro to GENIE double ratios. 
There is a degree of statistical limitation of the studies at this point, evident in the presence of bin-to-bin variations. To explore this point further, these studies can quickly and easily be extended to larger data sets to deal with statistics limitations. Additionally, the statistical uncertainties in the single and double ratios will also be investigated. With these, we can move on to quantify the degree to which the near-to-far extrapolation covers variations in the $\nu_{\mu}$ case - or does not cover variations in the $\overline{\nu_{\mu}}$ case. 

\section{Future Work}\label{sec:Future}

The above studies need to be furthered and expanded upon to successfully arrive at useful conclusions on ND configuration choice. 

\begin{itemize}


\item Extend studies to include LAr and GAr efficiency and acceptance information when available.
%\item Current uncertainties are invariant of target choice. Need to extend studies to include Calcium-40 target for FGT studies
\item Include a mapping of $E_{true}$ to $E_{reco}$ along with $y_{true}$ to $y_{reco}$ and investigate differences between configurations.

\end{itemize}


%\subsection{Ereco, yreco}
%An additional study to be commenced is the mapping from Ereco \& yreco into other kinematic variables (i.e. $Q^2$, $q_0$ vs. $q_3$), and the ability to similarly map model variations.
\appendix

%\subfile{Appendix_Neutron_Multi.tex}

%\subfile{Appendix_Nucleon_Multi.tex}

%\subfile{DUNE_appendix_Ereco_plots.tex}
%\subfile{DUNE_appendix_Q2_plots.tex}
\subfile{DUNE_appendix_q0_q3_plots.tex}


\begin{thebibliography}{7}

\bibitem{DUNE_CDR1}
The DUNE Collaboration
\textit{Long-Baseline Neutrino Facility (LBNF) and Deep Underground Neutrino Experiment (DUNE) Conceptual Design Report Volume 1: The LBNF and DUNE Projects}
arXiv:1601.05471v1

\bibitem{DUNE_review}
Maury Goodman
\textit{The Deep Underground Neutrino Experiment}
Advances in High Energy Physics, vol. 2015, Article ID 256351, 9 pages, 2015. doi:10.1155/2015/256351

\bibitem{GENIE}
Andreopoulos, C. \textit{et al}.
\textit{The GENIE Neutrino Monte Carlo Generator}.
Nucl.Instrum.Meth. A614 (2010) 87-104 arXiv:0905.2517 [hep-ph] FERMILAB-PUB-09-418-CD

\bibitem{NEUT}
Hayato, Yoshinari 
\textit{A neutrino interaction simulation program library NEUT}.
Acta Phys.Polon. B40 (2009) 2477-2489

\bibitem{NUWRO}
T. Golan, J.T. Sobczyk, J. Zmuda
\textit{NuWro: the Wrocław Monte Carlo Generator of Neutrino Interactions}.
Nuclear Physics B - Proceedings Supplements 229 (2012) 499

\bibitem{NUISANCE}
P. Stowell, C. Wret, C. Wilkinson, L. Pickering, S. Cartwright, Y. Hayato, K. Mahn, K.S. McFarland, J. Sobczyk, R. Terri, L. Thompson, M.O. Wascko, Y. Uchida
\textit{NUISANCE: a neutrino cross-section generator tuning and comparison framework}
arXiv:1612:07393v2

\bibitem{VALOR}
Andreopoulos, \textit{et al.}
\textit{VALOR DUNE Joint Oscillation and Systematics Constraint Fit}


\end{thebibliography}

\end{document}




