\setlength{\headheight}{15pt}
\documentclass[12pt]{article}
\usepackage{fancyhdr}
\lhead{}
\chead{}
\rhead{}
\renewcommand{\headrulewidth}{0pt}
\pagestyle{fancy}
\usepackage{graphicx}
\usepackage[top=2cm,bottom=3cm]{geometry}
\usepackage[svgnames]{xcolor}
\usepackage[colorlinks=true,linkcolor=DarkBlue,citecolor=DarkBlue]{hyperref}
\usepackage{xspace}
\usepackage{rotating}
\usepackage{units}
%\usepackage{subfig}
%\usepackage{amssymb, amsmath}
\usepackage{amsmath}
\usepackage{authblk}
\usepackage{lineno}
\usepackage{listings} 
\usepackage[normalem]{ulem}
%\usepackage{placeins}
\usepackage[section]{placeins}

\usepackage{SIunits}
\usepackage{hepunits}
\usepackage{hepparticles}
\usepackage{cancel}
\usepackage{hepnames}
\usepackage{epstopdf}
\usepackage{mathtools}
\usepackage{caption}
\usepackage[aboveskip=-10pt]{subcaption}
\usepackage[capitalise]{cleveref}
\usepackage{braket}
\usepackage{slashed}


\newcommand{\todo}[1]{{\color{red} TODO: #1}}
\newcommand\red[1]{{\color{red}#1}}
\newcommand{\ccpi}{CC1$\pi^0$\xspace}
\newcommand{\ccpis}{CC$\pi^0$\xspace}
\newcommand{\ccpip}{CC1$\pi^+$\xspace}
\newcommand{\ncpi}{NC1$\pi^0$\xspace}
\newcommand{\ccqe}{CCQE\xspace}
\newcommand{\mares}{\ensuremath{M_A^\mathrm{res}}\xspace}
\newcommand{\ppi}{\ensuremath{|\mathbf{p}_{\pi^0}|}\xspace}
\newcommand{\mb}{MiniBooNE\xspace}
\newcommand{\minerva}{MINER\ensuremath{\nu}A\xspace}
\newcommand{\neut}{\textsc{neut}\xspace}
\newcommand{\nuance}{\textsc{nuance}\xspace}
\newcommand{\tmu}{\ensuremath{T_{\mu}}\xspace}
\newcommand{\pmu}{\ensuremath{|\textbf{p}_{\mu}|}\xspace}
\newcommand{\cost}{\ensuremath{\cos{\theta_{\mu}}}\xspace}
\newcommand{\enu}{\ensuremath{E_{\nu}}\xspace}
\newcommand{\qq}{\ensuremath{Q^{2}}\xspace}
\newcommand{\qqqe}{\ensuremath{Q^{2}_{\textrm{QE}}}\xspace}
\newcommand{\pf}{\ensuremath{p_{F}}\xspace}
\newcommand{\eb}{\ensuremath{E_{b}}\xspace}
\newcommand{\carb}{C\ensuremath{^{12}}\xspace}
\newcommand{\oxy}{O\ensuremath{^{16}}\xspace}
\newcommand{\ie}{i.e.\xspace}
\newcommand{\eg}{e.g.\xspace}
\newcommand{\ma}{\ensuremath{M_{\textrm{A}}}\xspace}
\newcommand{\maqe}{\ensuremath{M_{\textrm{A}}^{\textrm{QE}}}\xspace}
\newcommand{\numu}{\Pnum}
\newcommand{\nue}{\Pnue}
\newcommand{\numubar}{\APnum}
\newcommand{\nuebar}{\APnue}
\newcommand{\enuqerfg}{\ensuremath{E^{\nu}_{\textrm{QE,RFG}}}\xspace}
\newcommand{\enuqe}{\ensuremath{E^{\nu}_{\textrm{QE}}}\xspace}
\newcommand{\chisq}{\ensuremath{\chi^{2}}\xspace}
\newcommand{\chisqmin}{\ensuremath{\chi^{2}_{\textrm{min}}}\xspace}
\newcommand{\chtwo}{CH\ensuremath{_{2}}\xspace}
\newcommand{\wroclaw}{Wroc{\l}aw\xspace}
\newcommand{\km}{\kilo\meter\xspace}
\newcommand{\m}{\meter\xspace}
\newcommand{\evsq}{\eV\ensuremath{^{2}}\xspace}
\newcommand{\POD}{P{\O}D\xspace}
\newcommand{\ecal}{ECal\xspace}
\newcommand{\ecals}{ECals\xspace}
\newcommand{\dsecal}{Ds-ECal\xspace}
\newcommand{\vol}[4]{\ensuremath{#1\times#2\times\unit{#3}{#4}}\xspace}
\newcommand{\area}[3]{\ensuremath{#1\times\unit{#2}{#3}}\xspace}
\newcommand{\pizero}{\pi^{0}\xspace}
\newcommand{\kg}{\kilo\gram\xspace}
\newcommand{\lep}{\ell}
\newcommand{\mnn}{multi-nucleon--neutrino\xspace}
\newcommand{\elt}{\ensuremath{E_{<}}\xspace}
\newcommand{\egt}{\ensuremath{E_{>}}\xspace}


\renewcommand\Im{\operatorname{Im}}

\graphicspath{{figures/}}

\newif\ifpdf
\ifx\pdfoutput\undefined
   \pdffalse
\else
   \pdfoutput=1
   \pdftrue
\fi
\ifpdf
   \usepackage{graphicx}
   \usepackage{epstopdf}
   %\DeclareGraphicsRule{.eps}{pdf}{.pdf}{`epstopdf #1}
   \pdfcompresslevel=9
\else
   \usepackage{graphicx}
\fi

\graphicspath{{figs/}}

\title{Current Status and Future Progess of DUNE ND studies}

\date{}
\begin{document}


\author[1]{Jake Calcutt}
\author[1]{Joshua Hignight}
\author[1]{Kendall Mahn}
\affil[1]{Michigan State University}
%\author[2]{Joshua Hignight}
%\author[3]{Kendall Mahn}


\maketitle
\thispagestyle{fancy}
%\linenumbers
%\begin{abstract}
%This report is a review of the current implementation of cross section related sources of systematic uncertainty for the DUNE Near Detector Taskforce (ND TF), charged with evaluating three possible near detector configurations. It identifies critical sources of systematic uncertainty, some of which are already covered, suggests studies to further improve the current uncertainty implementation, and summarizes future improvements to the systematic uncertainty outside the scope of the ND TF. 
%\end{abstract}
%A clear statement of what the committee deems to be an ideal (practical) scenario would be an excellent start. We can bring that to VALOR to determine what they can implement, and then ask the committee for some feedback on the limitations of the final VALOR parameterization to put in the report along with their full recommendations. The full simulation and analysis chain should not die with the task force, and the final report should include recommendations for what should be studied beyond the TF timeline.

\section{Overview}\label{sec:view}

%This document serves as a writeup detailing the current status of the DUNE ND studies and the progress that has been made so far, as well as a plan on how to further the studies in the near future. 

The Deep Underground Neutrino Experiment (DUNE) is a next-generation Long Baseline neutrino experiment aimed to achieve current scientific goals set out by the High Energy Physics community. It consists of both a Near and Far Detector separated by 1300km and standing in the NuMI beam created at Fermilab\cite{DUNE_CDR1}. While the design of the Liquid Argon Far Detector has been finalized, there is still ongoing effort in deciding the configuration of the Near Detector. The main near detector design includes a Fine-Grained Tracker (FGT) with possible inclusion of an upstream detector - being either a Liquid (LArTPC) or High Pressure Gaseous Argon TPC (GArTPC). DUNE's goals will require systematic uncertainties in the interaction model to be below the 2\% limit after a near-to-far extrapolation\cite{DUNE_review}.  The focus of the work described in this document is then to quantify the abilities of the standalone FGT detector and additional LAr/GAr TPC to achieve this limit in the near-to-far extrapolation. The sufficiency of current kinematic parameterization to handle model variations is also considered. 
%\begin{enumerate}
%\item Review the available material describing the handling of cross-section uncertainties within the DUNE Near Detector Task Force in the (\url{http://docs.dunescience.org:8080/cgi-bin/ShowDocument?docid=1291}{VALOR TN} sections 3 and 4 and the material presented at the 10am CT Thu July 14 ND Physics Working Group Meeting).
%\item Recommend changes within the existing framework that would better describe the current level of uncertainty in neutrino-nucleus interaction physics. These recommendations should be prioritized w.r.t. how crucial they are to the extraction of oscillation parameters (especially CP violation) and should take into account the limited timescale and manpower of the the NDTF (initial report Sept 2016, final report March 2017 ? about 0.5 FTE of cross-section resources).
%\item Make suggestions for studies that would help confirm the assumptions and/or results of the NDTF, should the time and manpower for these studies become available.
%\item  Make recommendations for a long term strategy for DUNE to study ND capabilities to i) measure neutrino-nucleus interactions and ii) constrain oscillation physics systematic uncertainties,  beyond the scope of the task force (task force charge can be found at this \url{https://docs.google.com/presentation/d16yY5CzRwo_243jpBzhVVeunvxYjIb7FzoS745RwURgw/edit#slide=id.p7}{link}.
%\end{enumerate}

\section{Methods}
As the analysis techniques for DUNE are developed, checks on the cross section model are necessary. Variations arise in the different handling of Final State Interaction (FSI) effects by Monte Carlo event generators, as well as choice of Near Detector configuration. Sets of neutrino and antineutrino events on Argon-40 are produced with Monte Carlo Generators - GENIE\cite{GENIE} version 2.10, using an RFG model, NEUT\cite{NEUT} version 5.3.6, using Nieves et. al RPA+2p2h/MEC $M_A$ = 1.01, and NuWro\cite{NUWRO} version 11, using LFG + RPA + Nieves et. al - according to the 2015 DUNE CDR fluxes. The $\nu_\mu$ Near Detector and Far Detector fluxes used in this work are showin in Figure~\ref{fig:dune_flux}.  The various data sets are passed through the NUISANCE\cite{NUISANCE} software to reduce the various outputs to a common format, in turn saving all final state particle information for each event. 
\begin{figure}[h]
\centering
\minipage{.5\textwidth}
\includegraphics[width=\linewidth]{Dune_Flux/numu_ND_flux.png}
\endminipage
\minipage{.5\textwidth}
\includegraphics[width=\linewidth]{Dune_Flux/numu_FD_flux.png}
\endminipage
\caption{$\nu_\mu$ flux at DUNE ND (Left) and FD (Right)}
\label{fig:dune_flux}
\end{figure}
\FloatBarrier

Each particle is then randomly accepted or rejected by throwing a random number and checking against the efficiency according to the particle and its momentum. Currently, only a robust description of the FGT efficiency is available. An example of the efficiency for protons in the FGT is given in Figure~\ref{fig:FGT_proton_effs}. Simple thresholds for protons are applied for the GAr - 100 MeV/c - and LAr ND and FD - 200 MeV/c. Currently, we do not have the efficiencies for $\mu$, $\pi^{+,-}$, and $\pi^{0}$ in the LAr and GAr, and so the detectors are assumed to be perfectly efficient (no rejections) to these particles. $\pi^0$ efficiency information is also currently missing in the FGT configuration, and so are assumed fully accepted.
\begin{figure}[h]
\centering
\minipage{.5\textwidth}
\includegraphics[width=\linewidth]{eff_plots/fgt_trkeff_proton.png}
\endminipage
\minipage{.5\textwidth}
\includegraphics[width=\linewidth]{eff_plots/fgt_pideff_proton.png}
\endminipage
\caption{Left: Tracking efficiency for protons in FGT. Right: PID efficiency for protons in FGT. Total efficiency for a given proton momentum is given by the product of the two efficiencies.}
\label{fig:FGT_proton_effs}
\end{figure}

%\newline

\section{Reconstructed Energy}\label{sec:Reco}
%\subsection{Final State, Reconstructed Energy}

A framework for investigating variations in reconstructed energy calculations between different generators and ND configurations has been developed. %However, because it is calculated only by summing the final state energy (generator truth information), this is largely incomplete, and it serves only as a starting point for more robust studies. The full studies will include detector efficiencies and kinematic cuts to fully display the differences between models and configurations. 

The current development of this work includes a calculation of the final state energy by summing the total energy from final state leptons and pions (all charges) and the kinetic energy of final state protons after passing efficiencies through the data sets. All neutrons are assumed undetectable. 
This can be summed up in Equation~\ref{eq:ereco}:
\begin{equation}
E_{reco} = \Sigma {E_{lep}} + \Sigma E_{\pi} + \Sigma (E_{prot} - M_{prot})
\label{eq:ereco}
\end{equation}
where $E_{lep}$ is the energy of the outgoing lepton, $E_{pi}$ is the energy of the charged pion, and $E_{prot}$ $(M_{prot})$ is the energy (momentum) of the proton.
%This is achieved by using the NUISANCE software package to reduce the output from various Monte Carlo neutrino event generators (including GENIE, NEUT, NUWRO, as well as the nuclear reaction and transport simulation software, GiBuu) into a common format that can easily be analyzed. 
%Work has begun to include tracking/PID efficiencies to accept or reject final state particles. So far, only the full information (barring $\pi^0$ efficiencies) for the FGT is available, but the inclusion of the LAr or GAr TPCs will be easily implemented. 

%\subsection{$E_\nu - E_{reco}$} 
\subsection{Difference from True Neutrino Energy}
\label{subsec:EDiff}
To investigate these variations, the difference between true and reconstructed neutrino energy from each generator - where NuWro and NEUT distributions have been normalized to GENIE - are plotted for each near detector configuration as well as the far detector, and for different reaction types. The reaction types considered are true-CCQE, true-2p2h, CC0$\pi$, CC1$\pi$, and CCOther. True-CCQE and true-2p2h are both defined as MC-level CCQE/2p2h interaction with 1 reconstructed lepton. CC0$\pi$ is defined as 0 $\pi^{\pm}$, 1 lepton, and any number of protons and $\pi^0$ after reconstruction. CC1$\pi$ is defined as 1 $\pi^{\pm}$, 1 lepton, and any number of protons and $\pi^0$ after reconstruction. Finally, CCOther is defined as any final state with 1 lepton and any number of hadrons reconstructed.

In $\nu_\mu$ mode: for all reaction types except CCOther, all three generators seem to have similar differences in all detector configurations. This is evident in Figure~\ref{fig:numu_Etrue_ereco_FGT_CCQE_and_CCOther} and the figures in appendix (ref appendix). For CCOther, all three generators appear to have a long tail extending to high discrepancy regions - well past that of CCQE for example - though NuWro has a higher amount of events in the $\Delta E$ = .5 GeV to 2 GeV region. 
\begin{figure}[h]
\centering
\minipage{.5\textwidth}
\includegraphics[width=\linewidth]{Ereco_Etrue/numu_FGT_CCQE.png}
\endminipage
\minipage{.5\textwidth}
\includegraphics[width=\linewidth]{Ereco_Etrue/numu_FGT_CCOther.png}
\endminipage
\caption{Difference between true and reconstructed Neutrino Energy in the FGT ND. CCQE events seem to have similar discrepancies, while CCOther mode has large tails in all three generators with NuWro having more events with higher discrepancy than the other two. }
\label{fig:numu_Etrue_ereco_FGT_CCQE_and_CCOther}
\end{figure}
\FloatBarrier

Meanwhile, for $\bar{\nu_\mu}$ CCQE, 2p2h, and CC1$\pi$ events NuWro consistently has $\Delta E$ closer to 0 than the other generators as shown in Figure~\ref{fig:numubar_Etrue_ereco_FGT_CCQE_2p2h_CC0Pi}. In CC1$\pi$ and CCOther NuWro again has more events with higher discrepancies, shown in Figure~\ref{fig:numubar_Etrue_ereco_FGT_CC1Pi_CCOther}. Additionally, a large difference in the distribution of events from NEUT and NuWro between the FGT and LAr FD, which is slightly better in the LAr ND, as can be seen in Figure~\ref{fig:numubar_Etrue_ereco_CCOther_diffs}.
\begin{figure}[h]
\centering
\minipage{.3\textwidth}
\includegraphics[width=\linewidth]{Ereco_Etrue/numubar_FGT_CCQE.png}
\endminipage
\minipage{.3\textwidth}
\includegraphics[width=\linewidth]{Ereco_Etrue/numubar_FGT_2p2h.png}
\endminipage
\minipage{.3\textwidth}
\includegraphics[width=\linewidth]{Ereco_Etrue/numubar_FGT_CC0Pi.png}
\endminipage
\caption{NuWro can be seen to have consistently less discrepancy in reconstructed energy than NEUT and GENIE in CCQE, 2p2h, and CCO$\pi$. }
\label{fig:numubar_Etrue_ereco_FGT_CCQE_2p2h_CC0Pi}
\end{figure}
\begin{figure}[h]
\centering
\minipage{.5\textwidth}
\includegraphics[width=\linewidth]{Ereco_Etrue/numubar_FGT_CC1Pi.png}
\endminipage
\minipage{.5\textwidth}
\includegraphics[width=\linewidth]{Ereco_Etrue/numubar_FGT_CCOther.png}
\endminipage
\caption{NuWro has higher discrepancy in reconstructed energy than NEUT and GENIE in both CC1$\pi$ and CCOther.}
\label{fig:numubar_Etrue_ereco_FGT_CC1Pi_CCOther}
\end{figure}
\begin{figure}[h]
\centering
\minipage{.3\textwidth}
\includegraphics[width=\linewidth]{Ereco_Etrue/numubar_FD_CCOther.png}
\endminipage
\minipage{.3\textwidth}
\includegraphics[width=\linewidth]{Ereco_Etrue/numubar_FGT_CCOther.png}
\endminipage
\minipage{.3\textwidth}
\includegraphics[width=\linewidth]{Ereco_Etrue/numubar_LAr_CCOther.png}
\endminipage
\caption{NuWro CCOther events have differences in shape between the Far Detector and LAr FD. The LAr ND matches more closely.}
\label{fig:numubar_Etrue_ereco_CCOther_diffs}
\end{figure}
\FloatBarrier

%As a first check in the ability to reconstruct neutrino energy, I began investigating the differences between the total final state energy and initial state energy. I've verified with Callum that the NUISANCE framework successfully treats the particle stacks from the various generators, and any differences come from how the different generators handle FSI. 



\subsection{Neutron Multiplicities \& Energy}
\label{sec:N_multiplicities_Energy}

Differences between models in the number of final state neutrons and the total energy into FS neutrons can largely affect reconstruction of neutrino energy. Large variations in reconstructed energy can arise due to missing energy caused by the inability to detect neutrons in the various models.  

To investigate this, we have looked at GENIE, NEUT, and NUWRO to see if the different models showed a large difference in the neutron energy and multiplicity.  
This was done for CCQE-like, CC1$\pi$, 2P2H, and everything else (``Other'') interactions and neutrinos as well as anti-neutrinos.  
In all cases, the generators agreed rather well with each other even though there were some differences in the neutron multiplicity. 
These differences only account for a small fraction of the events.
Figure~\ref{fig:Neutron_multi_2p2h_ND} shows an example of this for 2P2H neutrino events, while the other interaction modes can be found in Appendix~\ref{app:Neutron_Multiplicities}
  
\begin{figure}
\centering
\begin{subfigure}[b]{0.32\textwidth}
  \includegraphics[width=\textwidth]{nneutrons_v_total_ene/Nneutrons_Total_ENe_2p2h_GENIE_ND_numu.pdf}
\end{subfigure}
\begin{subfigure}[b]{0.32\textwidth}
  \includegraphics[width=\textwidth]{nneutrons_v_total_ene/Nneutrons_Total_ENe_2p2h_NEUT_ND_numu.pdf}
\end{subfigure}
\begin{subfigure}[b]{0.32\textwidth}
  \includegraphics[width=\textwidth]{nneutrons_v_total_ene/Nneutrons_Total_ENe_2p2h_NUWRO_ND_numu.pdf}
\end{subfigure}
\caption{The neutron multiplicity vs total neutron energy for 2P2H interactions for GENIE, NEUT, and NUWRO, respectively.  Even though they do show a different phase-space for the neutron multiplicity, they all agree where most of the energy lost to neutrons should be.  This is similar for other interaction types as well.}
\label{fig:Neutron_multi_2p2h_ND}
\end{figure}
Further more, the difference in the multiplicities between the generators becomes irrelevant after a ND to FD extraction, as can be seen in Figure~\ref{fig:Neutron_multi_2p2h_ND_FD}.
Here, the region where the most energy is lost agrees very well between the ND and FD for all generators and interaction types.
The areas with low statistics do show a disagreement, but very few events fall into this area.

\begin{figure}
\centering
\begin{subfigure}[b]{0.32\textwidth}
  \includegraphics[width=\textwidth]{nneutrons_v_total_ene/Nneutrons_Total_ENe_2p2h_GENIE_ND_FD_numu_norm.pdf}
\end{subfigure}
\begin{subfigure}[b]{0.32\textwidth}
  \includegraphics[width=\textwidth]{nneutrons_v_total_ene/Nneutrons_Total_ENe_2p2h_NEUT_ND_FD_numu_norm.pdf}
\end{subfigure}
\begin{subfigure}[b]{0.32\textwidth}
  \includegraphics[width=\textwidth]{nneutrons_v_total_ene/Nneutrons_Total_ENe_2p2h_NUWRO_ND_FD_numu_norm.pdf}
\end{subfigure}
\caption{The ratio of the ND to the FD for neutron multiplicity vs total neutron energy for 2P2H interactions for GENIE, NEUT, and NUWRO, respectively.  In the area where the largest amount of energy is lost to neutrons, low multiplicity and low energy, the agreement between the ND and FD is almost perfect.} 
\label{fig:Neutron_multi_2p2h_ND_FD}
\end{figure}
Care should still be taken when calculating the total neutrino of the event if only a calorimetric approach is used, as is the case in this document.
This is because as much as 50\% of the energy can be taken away by the neutron in CCQE-like events.
Even for 2P2H events, as shown in Figure~\ref{fig:Neutron_multi_ene_enu_2p2h_ND}, it can be as much as 30\%.
The other interaction modes and anti-neutrino events can also be found in Appendix~\ref{app:Neutron_Multiplicities} and have a smaller fraction.

\begin{figure}
\centering
\begin{subfigure}[b]{0.32\textwidth}
  \includegraphics[width=\textwidth]{nneutrons_ene_enu/Nneutrons_Enu_true_2p2h_GENIE_ND_numu_norm.pdf}
\end{subfigure}
\begin{subfigure}[b]{0.32\textwidth}
  \includegraphics[width=\textwidth]{nneutrons_ene_enu/Nneutrons_Enu_true_2p2h_NEUT_ND_numu_norm.pdf}
\end{subfigure}
\begin{subfigure}[b]{0.32\textwidth}
  \includegraphics[width=\textwidth]{nneutrons_ene_enu/Nneutrons_Enu_true_2p2h_NUWRO_ND_numu_norm.pdf}
\end{subfigure}
\caption{Neutron multiplicity vs total neutron energy divide by the neutrino energy for 2P2H interactions from GENIE, NEUT, and NUWRO, respectively.  For low multiplicity, the neutron carries away a significant fraction of the neutrino energy.} 
\label{fig:Neutron_multi_ene_enu_2p2h_ND}
\end{figure}
\FloatBarrier

 Additionally, investigations in the ability for errors in GENIE to cover the differences between models have been started. 

\subsection{Particle Multiplicity and Momentum}
\label{sec:Particle_multiplicities_Energy}

Nucleon multiplicity and momentum distributions offer similar information as do N vs. E distributions, while specifically looking at protons and charged pions along with detector thresholds can enlighten the ability of the detectors' reconstruction capabilities. For these studies, the 3-momentum of the final state (after efficiencies) protons or charged pions are summed. The magnitude is then plotted against the multiplicity for the specific particle type. 
\begin{figure}[h]
\minipage{.3\textwidth}
\includegraphics[width=\linewidth]{eff_N_P/FGT/protons/ratios/CCQE_NEUT_GENIE_numu_near_N_P.png}
\endminipage
\minipage{.3\textwidth}
\includegraphics[width=\linewidth]{eff_N_P/FGT/protons/ratios/CCQE_NEUT_GENIE_numu_far_N_P.png}
\endminipage
\minipage{.3\textwidth}
\includegraphics[width=\linewidth]{eff_N_P/FGT/protons/ratios/CCQE_NEUT_GENIE_numu_NF_N_P.png}
\endminipage
\caption{Nprotons vs. P protons distributed events using DUNE flux. Left: Ratio of NEUT to GENIE output at ND with FGT efficiencies, Mid: Ratio of NEUT to GENIE output at FD with LAr efficiencies, Right: Double ratio of NEUT to GENIE, Near to Far}
\end{figure}
\FloatBarrier
%\begin{equation}
%|\vec{P_{total,i}}| = |\Sigma \vec{P_i}|
%\end{equation}

\section{Parameterization}
The VALOR group\cite{VALOR} has led multiple T2K oscillation analyses and has contributed to most published T2K oscillation papers(I TOOK THIS ALMOST WORD FOR WORD FROM THE VALOR DOC BUT IDK WHAT TO WRITE FOR IT. IF CITING IT IS NOT GOOD ENOUGH I NEED TO CHANGE THIS). It also has contributed to to optimisation studies for DUNE, this being a source of motivation for this work. 

The group uses MC templates to map between reconstructed and true information for various reactions. Currently, $Q^2$ parameterizations have been suggested by VALOR for various MC templates and are defined for both neutrinos and antineutrinos as: 
\begin{itemize}
\item CCQE with bins \{0 - 0.20, 0.20 - 0.55, $>$0.55\} $\textrm{GeV}^2$
\item CC1$\pi^{\pm}$ with bins \{0 - 0.30, 0.30 - 0.80, $>$0.80\} $\textrm{GeV}^2$
\item CC1$\pi^0$ with bins \{0 - 0.35, 0.35 - 0.90, $>$0.90\} $\textrm{GeV}^2$
\item CCOther with 1 $Q^2$ bin
\end{itemize}

The following portion of this work is devoted to investigation the sufficiency of these $Q^2$ parameterizations in treating model FSI uncertainties in the various detector configurations.

\subsection{$Q^2$ studies}
An obvious first step the investigation of $Q^2$ parameterization sufficiency is to consider the relative change between the models and this change as it is extrapolated between the Near and Far Detector. To do this, events are created by all three models at the Near and Far Detector and distributed according to $Q^2$. This is done separately both without efficiencies, where $Q^2$ is the MC-level truth information, and with efficiencies for all three ND configurations and the FD, where $Q^2$ is now a reconstructed quantity. Ratios of NEUT either or NuWro to GENIE - referred to as 'single ratios' and with NEUT and NuWro both normalized to GENIE - at the Near and Far Detectors offer insight into model variations and how these depend on ND configuration. The effects of a near-to-far extrapolation and its ability to effectively 'wash out' these variations are enlightened by taking a 'double ratio' of Near and Far single ratios for a given MC combination. To be explicit and to avoid confusion, these are defined in Equations ~\ref{eq:single_ratio} and ~\ref{eq:double_ratio}, where 'Other MC' refers to either NEUT or NuWro and 'Near' can be replaced by 'Far' in the single ratio. This is done for both $\nu_{\mu}$ and $\bar{\nu_{\mu}}$, and for true-CCQE, true-2p2h, CC0$\pi^{\pm}$, CC1$\pi^{\pm}$, and CCOther as defined in Section~\ref{subsec:EDiff}

\begin{equation}
\label{eq:single_ratio}
\textrm{ND Single Ratio} = \frac{\textrm{(Other MC @ Near Detector)}}{\textrm{(GENIE @ Near Detector)}}
\end{equation}

\begin{equation}
\label{eq:double_ratio}
\textrm{Double Ratio} = \frac{\textrm{(ND Single Ratio)}}{\textrm{(FD Single Ratio)}}
\end{equation}

For $\nu_{\mu}$ CCQE events, single ratio distributions have general 'unflatness', but the double ratio distributions remain relatively flat and close to 1 for distributions without efficiencies, shown in Figure~\ref{fig:Q2_ccqe_no_eff}. For double ratios with efficiencies applied, though variations are greater in the higher end of these distributions, they do appear centered close to or near 1 as seen in Figure~\ref{fig:Q2_ccqe_FGT_eff}. 
\begin{figure}[h]
\minipage{.5\textwidth}
\includegraphics[width=\linewidth]{Q2/nominal/ratios/CCQE_NEUT_GENIE_numu_near_Q2.png}
\endminipage
\minipage{.5\textwidth}
\includegraphics[width=\linewidth]{Q2/nominal/ratios/CCQE_NEUT_GENIE_numu_NF_Q2.png}
\endminipage
\caption{$Q^2$ distributed events using DUNE flux, no efficiencies applied. Left: Ratio of NEUT to GENIE output at ND, Right: Double ratio of NEUT to GENIE, Near to Far. Of note is the relative flatness throughout the double ratio compared to the single ratio at the ND.}
\label{fig:Q2_ccqe_no_eff}
\end{figure}
\begin{figure}[h]
\minipage{.5\textwidth}
\includegraphics[width=\linewidth]{eff_Q2/FGT/ratios/CCQE_NEUT_GENIE_numu_near_Q2.png}
\endminipage
\minipage{.5\textwidth}
\includegraphics[width=\linewidth]{eff_Q2/FGT/ratios/CCQE_NEUT_GENIE_numu_NF_Q2.png}
\endminipage
\caption{$Q^2$ distributed events using DUNE flux, FGT efficiencies applied to ND and LAr effiencies applied to FD. Left: Ratio of NEUT to GENIE output at ND (high $Q^2$ region out of bounds of plot), Right: Double ratio of NEUT to GENIE, Near to Far. More variations do arise in the higher regions of the double ratio, but still appear centered close to or around 1.}
\label{fig:Q2_ccqe_FGT_eff}
\end{figure}
\FloatBarrier

However, for $\bar{\nu_{\mu}}$ CCQE events, this flatness is generally lost in the double ratios - without efficiencies, as well as all three configurations - for both NEUT and NuWro, but being particularly worse in NuWro. This is displayed in Figure~\ref{fig:Q2_ccqe_bar}. The large variations in $Q^2$ are similar to and possibly arise from NuWro's trend towards larger differences in true and reconstructed neutrino energy as seen in Section~\ref{subsec:EDiff}. This trend of larger variations in $\bar{\nu_{\mu}}$ as opposed to $\nu_{\mu}$ events continues in all reaction types except for 2p2h and can be seen in Appendix~\ref{} where all plots are stored.
\begin{figure}[h]
\minipage{.25\textwidth}
\includegraphics[width=\linewidth]{Q2/nominal/ratios/CCQE_NEUT_GENIE_numubar_NF_Q2.png}
\endminipage
\minipage{.25\textwidth}
\includegraphics[width=\linewidth]{Q2/nominal/ratios/CCQE_NuWro_GENIE_numubar_NF_Q2.png}
\endminipage
\minipage{.25\textwidth}
\includegraphics[width=\linewidth]{eff_Q2/FGT/ratios/CCQE_NEUT_GENIE_numubar_NF_Q2.png}
\endminipage
\minipage{.25\textwidth}
\includegraphics[width=\linewidth]{eff_Q2/FGT/ratios/CCQE_NuWro_GENIE_numubar_NF_Q2.png}
\endminipage
\caption{$Q^2$ $\bar{\nu_{\mu}}$ CCQE events using DUNE flux. Left to right: NEUT to GENIE double ratio no efficiencies, NuWro to GENIE double ratio no efficiencies, NEUT to GENIE double ratio FGT efficiencies, NuWro to GENIE double ratio FGT efficiencies. Large amounts of variations throughout all distributions, particularly in NuWro. }
\label{fig:Q2_ccqe_bar}
\end{figure}
\FloatBarrier

For 2p2h events, large variations in $\bar{\nu_{\mu}}$ ratios are present, similar to the other reaction modes. See Appendix~\ref{} for these. However, these large variations also occur in $\nu_{\mu}$ ratios, though these appear to be from low statistics. Shown in Figure~\ref{fig:Q2_2p2h}, one can see a cut-off in $Q^2$ bins $> \textrm{1.5} \textrm{GeV}^2$ in the NEUT,GENIE double ratio without efficiencies coming from a lack of events. Some lower $Q^2$ events then migrate into these bins after efficiencies are applied, resulting in relatively large error bars. For NuWro, though there are no cut-offs, the statistics are nonetheless quite low and result in large errors.

\begin{figure}[h]
\minipage{.25\textwidth}
\includegraphics[width=\linewidth]{Q2/nominal/ratios/2p2h_NEUT_GENIE_numubar_NF_Q2.png}
\endminipage
\minipage{.25\textwidth}
\includegraphics[width=\linewidth]{Q2/nominal/ratios/2p2h_NuWro_GENIE_numubar_NF_Q2.png}
\endminipage
\minipage{.25\textwidth}
\includegraphics[width=\linewidth]{eff_Q2/FGT/ratios/2p2h_NEUT_GENIE_numubar_NF_Q2.png}
\endminipage
\minipage{.25\textwidth}
\includegraphics[width=\linewidth]{eff_Q2/FGT/ratios/2p2h_NuWro_GENIE_numubar_NF_Q2.png}
\endminipage
\caption{$Q^2$ $\nu_{\mu}$ 2p2h events using DUNE flux. Left to right: NEUT to GENIE double ratio no efficiencies, NuWro to GENIE double ratio no efficiencies, NEUT to GENIE double ratio FGT efficiencies, NuWro to GENIE double ratio FGT efficiencies. Large amounts of variations resulting from low statistics - if any - in the higher end of all distributions.}
\label{fig:Q2_2p2h}
\end{figure}
\FloatBarrier


\subsection{$q_0 \textrm{vs.} q_3$ studies}

In comparison to the purely $Q^2$ parameterization, a simple $q_0 vs. q_3$ parameterization was also considered. One concern is the possibility of inconsistency between the two parameterizations in variations between models. 
%\newline
\begin{figure}[h]
\minipage{.5\textwidth}
\includegraphics[width=\linewidth]{q0_q3/nominal/ratios/CCQE_NEUT_GENIE_numu_near_q3_q0.png}
\endminipage
\minipage{.5\textwidth}
\includegraphics[width=\linewidth]{q0_q3/nominal/ratios/CCQE_NEUT_GENIE_numu_NF_q3_q0.png}
\endminipage
\caption{q3 vs. q0 distributed events using DUNE flux. Left: Ratio of NEUT to GENIE output at ND, Right: Double ratio of NEUT to GENIE, Near to Far}
\end{figure}
%\newline
\begin{figure}[h]
\minipage{.5\textwidth}
\includegraphics[width=\linewidth]{eff_q0_q3/FGT/ratios/CCQE_NEUT_GENIE_numu_near_q3_q0.png}
\endminipage
\minipage{.5\textwidth}
\includegraphics[width=\linewidth]{eff_q0_q3/FGT/ratios/CCQE_NEUT_GENIE_numu_NF_q3_q0.png}
\endminipage
\caption{q3 vs. q0 distributed events using DUNE flux, FGT efficiencies applied to ND and LAr effiencies applied to FD. Left: Ratio of NEUT to GENIE output at ND, Right: Double ratio of NEUT to GENIE, Near to Far.}
\end{figure}
\FloatBarrier

\subsection{Nucleon multiplicity vs. W}

Differences in mapping from $E_{reco}$ to true variables can arise from shape differences in nucleon multiplicities vs. W distributions. 
These distributions can also show where in the phase space most of the events shown in Section~\ref{sec:N_multiplicities_Energy} and Section~\ref{sec:Particle_multiplicities_Energy} exist.  A first look of this is shown in this section.

\subsubsection{Neutrons vs W}

Apart from the difference in neutron multiplicity discussed in Section~\ref{sec:N_multiplicities_Energy}, the exact phase space of these event in the neutron/W plane is slightly different.  
This is most pronounced for 2P2H events, as can be seen in Figure~\ref{Neutron_w_2p2h_ND}. 
The width of NEUT's peak is broader then GENIE's or NUWRO's while NUWRO allows for much higher W's.

\begin{figure}
\centering
\begin{subfigure}[b]{0.32\textwidth}
  \includegraphics[width=\textwidth]{nneutrons_w/Nneutrons_W_nuc_rest_2p2h_GENIE_ND_numu.pdf}
\end{subfigure}
\begin{subfigure}[b]{0.32\textwidth}
  \includegraphics[width=\textwidth]{nneutrons_w/Nneutrons_W_nuc_rest_2p2h_NEUT_ND_numu.pdf}
\end{subfigure}
\begin{subfigure}[b]{0.32\textwidth}
  \includegraphics[width=\textwidth]{nneutrons_w/Nneutrons_W_nuc_rest_2p2h_NUWRO_ND_numu.pdf}
\end{subfigure}
\caption{The neutron multiplicity vs W for 2P2H interactions from GENIE, NEUT, and NUWRO, respectively.  It can be seen that all three event generators have the peak of the distribution at the same place, but NEUT's peak is much broader while NUWRO allows for much higher W's.}
\label{fig:Neutron_w_2p2h_ND}
\end{figure}

However, Figure~\ref{fig:Neutron_w_2p2h_ND_FD} shows that the differences are not so important if a near to far extrapolation is used, as both the ND and FD have simular responses.
To see if the different models could have a larger affect on the physics results, we have taken a double ratio of the ND/FD and the different generators to GENIE.  
Interestingly, Figure~\ref{fig:Neutron_w_2p2h_ND_FD_GENIE} indicates that there would not be large affect if a few perecent difference between the models is acceptable.

\begin{figure}
\centering
\begin{subfigure}[b]{0.32\textwidth}
  \includegraphics[width=\textwidth]{nneutrons_w/Nneutrons_W_nuc_rest_2p2h_GENIE_ND_FD_numu_norm.pdf}
\end{subfigure}
\begin{subfigure}[b]{0.32\textwidth}
  \includegraphics[width=\textwidth]{nneutrons_w/Nneutrons_W_nuc_rest_2p2h_NEUT_ND_FD_numu_norm.pdf}
\end{subfigure}
\begin{subfigure}[b]{0.32\textwidth}
  \includegraphics[width=\textwidth]{nneutrons_w/Nneutrons_W_nuc_rest_2p2h_NUWRO_ND_FD_numu_norm.pdf}
\end{subfigure}
\caption{The ND/FD ratio of neutron multiplicity vs W for 2P2H interactions from GENIE, NEUT, and NUWRO, respectively.  The effects of the different phase space seen in Figure~\ref{fig:Neutron_w_2p2h_ND} is not great if a ND ro FD extropolation is used.  This can be seen by the ratio being very close to one at the peak of Figure~\ref{fig:Neutron_w_2p2h_ND} distribution.}
\label{fig:Neutron_w_2p2h_ND_FD}
\end{figure}

These results are also true for protons and pions, though pions have a much lower multiplicity.  
Those results, along with other interactions can be found in Appendix~\ref{ap:Nucleon_W}.

\begin{figure}
\centering
\begin{subfigure}[b]{0.32\textwidth}
  \includegraphics[width=\textwidth]{nneutrons_w/Nneutrons_W_nuc_rest_2p2h_GENIE_NEUT_ND_FD_numu_norm.pdf}
\end{subfigure}
\begin{subfigure}[b]{0.32\textwidth}
  \includegraphics[width=\textwidth]{nneutrons_w/Nneutrons_W_nuc_rest_2p2h_GENIE_NUWRO_ND_FD_numu_norm.pdf}
\end{subfigure}
\caption{The double ratio of ND/FD and the event generators to GENIE of the neutron multiplicity vs W for 2P2H interactions.  Despite the differences seen in the W distributions, the double ratio shows suprisingly good agreement between the generators, indicating that the different models should not have a large impact on the final results.}
\label{fig:Neutron_w_2p2h_ND_FD_GENIE}
\end{figure}

\subsubsection{Protons vs W}

In the last section we saw a differnce in the phase space between the different generators.  
Now we will show an instance where the phase space is almost the same but there is a large difference between the ND and FD.  
Though here we are only showing the results for CC1$\pi$ via protons, the results are the same for CC-Other (anything not CCQE, CC1$\pi$, or 2P2H), and neutrons along with pions.
Figure~\ref{fig:Proton_w_res_ND} shows that GENIE, NEUT, and NUWRO have very similar distributions.

\begin{figure}
\centering
\begin{subfigure}[b]{0.32\textwidth}
  \includegraphics[width=\textwidth]{nprotons_w/Nprotons_W_nuc_rest_res_GENIE_ND_numu.pdf}
\end{subfigure}
\begin{subfigure}[b]{0.32\textwidth}
  \includegraphics[width=\textwidth]{nprotons_w/Nprotons_W_nuc_rest_res_NEUT_ND_numu.pdf}
\end{subfigure}
\begin{subfigure}[b]{0.32\textwidth}
  \includegraphics[width=\textwidth]{nprotons_w/Nprotons_W_nuc_rest_res_NUWRO_ND_numu.pdf}
\end{subfigure}
\caption{The proton multiplicity vs W for CC1$\pi$ interactions from GENIE, NEUT, and NUWRO, respectively.  It can be seen that all three event generators have the peak of their distribution at the same place and are almost identical}
\label{fig:Proton_w_res_ND}
\end{figure}

If a near to far detector extrapolation is used, however, there are much larger differences in region of most interest. 
Figure~\ref{fig:Proton_w_res_ND_FD} shows that there is a difference between the two detectors between 20\% and 40\% at the peak of the distribution.  
There is also a clear difference in lower W values compared to higher W values. 


\begin{figure}
\centering
\begin{subfigure}[b]{0.32\textwidth}
  \includegraphics[width=\textwidth]{nprotons_w/Nprotons_W_nuc_rest_res_GENIE_ND_FD_numu_norm.pdf}
\end{subfigure}
\begin{subfigure}[b]{0.32\textwidth}
  \includegraphics[width=\textwidth]{nprotons_w/Nprotons_W_nuc_rest_res_NEUT_ND_FD_numu_norm.pdf}
\end{subfigure}
\begin{subfigure}[b]{0.32\textwidth}
  \includegraphics[width=\textwidth]{nprotons_w/Nprotons_W_nuc_rest_res_NUWRO_ND_FD_numu_norm.pdf}
\end{subfigure}
\caption{The ND/FD ratio of the proton multiplicity vs W for CC1$\pi$ interactions from GENIE, NEUT, and NUWRO, respectively.  Unlike with 2P2H events shown for neutrons, there are clear differences between the ND and FD.  At the peak of the distribution this is between 20\% and 40\%.  Furthermore, there is a clear difference between W values below 2000 and ones above it.}
\label{fig:Proton_w_res_ND_FD}
\end{figure}

We can again see if the different models could have a larger affect on the physics results, by taking the double ratio as we did before.
Simular to before, Figure~\ref{fig:Proton_w_res_ND_FD_GENIE} indicates that there would not be large affect if a few perecent difference between the models is acceptable.
In the region of interest, the difference between GENIE and NEUT is negligable while it is small compared to NUWRO.


\begin{figure}
\centering
\begin{subfigure}[b]{0.32\textwidth}
  \includegraphics[width=\textwidth]{nprotons_w/Nprotons_W_nuc_rest_res_GENIE_NEUT_ND_FD_numu_norm.pdf}
\end{subfigure}
\begin{subfigure}[b]{0.32\textwidth}
  \includegraphics[width=\textwidth]{nprotons_w/Nprotons_W_nuc_rest_res_GENIE_NUWRO_ND_FD_numu_norm.pdf}
\end{subfigure}
\caption{The double ratio of ND/FD and the event generators to GENIE of the proton multiplicity vs W for CC1$\pi$ interactions.  Despite the differences seen in the single ratio, the double ratio shows suprisingly good agreement between the generators}
\label{fig:Proton_w_res_ND_FD_GENIE}
\end{figure}



\section{Future Work}\label{sec:Future}

The above studies need to be furthered and expanded upon to successfully arrive at useful conclusions on ND configuration choice. 

\begin{itemize}


\item Extend studies to include LAr and GAr efficiency and acceptance information when available.
%\item Current uncertainties are invariant of target choice. Need to extend studies to include Calcium-40 target for FGT studies
\item Include a mapping of $E_{true}$ to $E_{reco}$ along with $y_{true}$ to $y_{reco}$ and investigate differences between configurations.

\end{itemize}


%\subsection{Ereco, yreco}
%An additional study to be commenced is the mapping from Ereco \& yreco into other kinematic variables (i.e. $Q^2$, $q_0$ vs. $q_3$), and the ability to similarly map model variations.


\begin{thebibliography}{1}

\bibitem{GENIE}
Andreopoulos, C. \textit{et al}.
\textit{The GENIE Neutrino Monte Carlo Generator}.
Nucl.Instrum.Meth. A614 (2010) 87-104 arXiv:0905.2517 [hep-ph] FERMILAB-PUB-09-418-CD

\bibitem{NEUT}
Hayato, Yoshinari 
\textit{A neutrino interaction simulation program library NEUT}.
Acta Phys.Polon. B40 (2009) 2477-2489

\bibitem{NUWRO}
T. Golan, J.T. Sobczyk, J. Zmuda
\textit{NuWro: the Wrocław Monte Carlo Generator of Neutrino Interactions}.
Nuclear Physics B - Proceedings Supplements 229 (2012) 499

\bibitem{NUISANCE}
P. Stowell, C. Wret, C. Wilkinson, L. Pickering, S. Cartwright, Y. Hayato, K. Mahn, K.S. McFarland, J. Sobczyk, R. Terri, L. Thompson, M.O. Wascko, Y. Uchida
\textit{NUISANCE: a neutrino cross-section generator tuning and comparison framework}
arXiv:1612:07393v2

\bibitem{DUNE_CDR1}
The DUNE Collaboration
\textit{Long-Baseline Neutrino Facility (LBNF) and Deep Underground Neutrino Experiment (DUNE) Conceptual Design Report Volume 1: The LBNF and DUNE Projects}
arXiv:1601.05471v1

\bibitem{DUNE_review}
Maury Goodman
\textit{The Deep Underground Neutrino Experiment}
Advances in High Energy Physics, vol. 2015, Article ID 256351, 9 pages, 2015. doi:10.1155/2015/256351

\bibitem{VALOR}
Andreopoulos, \textit{et al.}
\textit{VALOR DUNE Joint Oscillation and Systematics Constraint Fit}


\end{thebibliography}

\end{document}




